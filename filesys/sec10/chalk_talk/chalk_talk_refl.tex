\documentclass{article}

% Packages
\usepackage{fullpage}
\usepackage{graphicx}
\usepackage{xcolor}

% Macros
\newcommand{\ans}[1]{\color{blue} #1 \color{black}}

\begin{document}

    % TITLE SECTION

    \begin{center}
        ************************************ \\
        SEC10 Educational Talk, Self-Assessment \\
        Edward Speer \\
        \today \\
        ************************************
    \end{center}

    \noindent \emph{Q: Elaborate on two (2) or more things you witnessed in others'
        presentations you thought were effective in their presentations and you
        would like to incorporate into or improve in your own talks/talk style.}

    \begin{enumerate}
    
        \item In several of the other talks, I noticed there was an element of
              suspense of sorts, where something that was setup in the start of
              the talk was eventually revealed at the end. This is a good
              strategy to keep people engaged until the end to get their payoff,
              and I would like to incorporate this into my own talks.
        
        \item I noticed that some of the talks connected the topic to something
              more personal for the speaker, which made the talk more engaging.
              It helped make me as an audience member care about the topic of
              the talk when I wouldn't have otherwise. This isn't really
              something I usually strive for when teaching in a chalk talk
              scenario, because typically the people in that scenario are trying
              to learn something specific and therefore already care (think of
              TAing a class or tutoring). However, I think it would be a good
              the setting for this talk was more like a lecture, where the
              audience might not be as engaged as they would be in a smaller
              setting, and therefore this strategy would be more effective
              and helpful.

    \end{enumerate}

\end{document}
