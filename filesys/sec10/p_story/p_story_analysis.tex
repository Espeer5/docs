\documentclass{article}

% Packages
\usepackage{fullpage}
\usepackage{graphicx}
\usepackage{xcolor}

% Macros
\newcommand{\ans}[1]{\color{blue} #1 \color{black}}

\begin{document}

    % TITLE SECTION

    \begin{center}
        ************************************ \\
        SEC10 Personal Story, Self-Analysis \\
        Edward Speer \\
        \today \\
        ************************************
    \end{center}

    \begin{enumerate}
    
        \item \emph{\textbf{Insert a small version of the image} used in your
              Personal Story presentation and then write a very short
              description of the story.}

            \begin{center}
                \includegraphics[width=0.5\textwidth]{img.png}
            \end{center}

            \ans{
                In my personal story, I shared a woeful tale of sticking myself
                with an acid-soaked needle in Chem3X lab after pulling an
                all-nighter. The image above shows the experiment set-up which
                led to the disaster.}\hfill \\

        \item \emph{\textbf{Write a snappy one-sentence synopsis} of your
              Personal Story presentation.}

            \ans{A fateful mishap with an acid-covered needle in Chem3X lab once
                taught me a painful lesson about the importance of sleep.}
                \hfill \\

        \item \emph{What would you say was the \textbf{theme} or
            \textbf{central idea} of your story?}

            \ans{Prioritizing one's health and well-being is crucial to both
                academic success and personal safety.} \hfill \\

        \item \emph{What \textbf{story genre(s)} do you identify having used
              consciously or unconsciously in your story?}
        
            \ans{
                \begin{itemize}
                    \item Conscious: Comedy, Cautionary tale
                    \item Unconscious: Tragedy, Gore
                \end{itemize}
                \hfill
            }

        \pagebreak

        \item \emph{What \textbf{story structure} did you employ in your story?
              How did you build/structure it?}
            
            \ans{
                My story was structured as a 3-act narrative. The first act was
                to set the scene: explain why I was at chem3X lab after pulling
                an all-nighter. Next, I introduced the experiment we were
                performing, heavily foreshadowing a disaster by focusing on the
                acid-soaked needle used. Finally, in the third act, there is a 
                crisis moment where I stick myself with the needle. The third
                act ends with a resolution of sorts, where I explain there was
                no lasting damage and reflect on the lesson learned. 
            }

        \item \emph{a: What \textbf{story devices} did you employ in your
                       story? \\
                    b: What story devices did you see in other presentations
                       that you could have incorporated into your own story?}

            \ans{
                \begin{enumerate}
                    \item I heavily used foreshadowing to imply danger and lead
                          up to a story climax. In my story, I began by 
                          implying that the combination of CS3 and Chem3X
                          brought some harm onto me. I then hyperfocused on the
                          needle and acid throughout the body. Both of these 
                          foreshadow the climax. I also used comedy to lighten
                          the mood of the story and engage the audience.
                    \item On reflection, I could have made use of very dramatic 
                          questions to enhance the humorous tone of the talk and
                          draw more audience engagement.
                \end{enumerate}
                \hfill
            }

        \item \emph{What things \textbf{worked best} in your \textbf{Personal
              Story} presentation?}
            
            \ans{
                I think the comedy worked out well, I got some decent laughs and
                the audience seemed engaged. Similarly, asking questions to the
                audience was a good way to draw the engagement as well.
            } \hfill \\

        \item \emph{What things \textbf{needed more work} or
              \textbf{could be improved} in your \textbf{Personal Story}
              presentation?}

            \ans{
                I may have been tied too much to the script according to the
                professor's feedback. I don't think I scanned the audience very 
                much, and I could have done a better job of writing my story to
                have more of an arc. I was focused on including comedy much more
                than scupting the narrative.
            } \hfill \\

        \item \emph{What thing(s) did you believe you had \textbf{under control}
              before the presentation, but ended up \textbf{surprising} you in
              the actual delivery? And, looking back on it, why do you think
              that happened?}

            \ans{
                I didn't expect to be nervous at all for this presentation. I
                have done public speaking before with no nerves or issues, and
                was surprised to find myself feeling nervous for no reason
                during this talk. I think it was because the exercise was so
                focused on the speaking itself. Normally I am speaking to convey
                information that I have some passion for, and not for the sake
                of speaking itself. Being so hyper focused on my body
                positioning and speaking mannerisms added some element of
                self-consciousness that was different and strange.
            } \hfill \\

        \item \emph{Of the many things listed above, which do you feel you need
              \textbf{external help} with the most?}

            \ans{
                I don't feel that I need external help. I think I can resolve
                these concerns by practicing more, as will happen throughout the
                remainder of this course.
            }

    \end{enumerate}

\end{document}
