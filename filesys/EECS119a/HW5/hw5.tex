\documentclass{article}

% Packages
\usepackage{fullpage}
\usepackage{graphicx}
\usepackage{verbatim}
\usepackage{pdflscape}

% Macros
% INSERT HW NUM HERE
\newcommand{\HWNUM}{5}
\newcommand{\fig}[2]{
    \begin{center}
        \includegraphics[scale=#1]{figs/#2}
    \end{center}
}

\begin{document}

    % TITLE SECTION

    \begin{center}
        ************************************ \\
        EE/CS 119A Homework \HWNUM \\
        Edward Speer \\
        \today \\
        ************************************
    \end{center}

    % PROBLEMS
    \begin{enumerate}
    
        \item \textbf{Problem 1}: \\
            Since CLK is in the sensitivity list, we need to update $x$ to
            reflect the value of $a$ on both the rising and falling edges of 
            the clock. We therefore need 2 DFFs to detect each edge and store
            the value of $a$ at that time. We then know which DFF should be
            output by the value of the CLK signal. If the clock is high, we
            output the value of the DFF storing the value of $a$ on the rising
            edge and vice versa.
            \fig{.4}{p1.jpg}

        \item \textbf{Problem 2}: \\
            Since the clock is in the sensitivity list, but we only update $x$
            when the clock is high, we only need a single DFF to detect the
            rising edge of the clock. When we do detect the rising edge, we
            simply check if $b$ is high, and either pass through $a$ or keep $x$
            the same accordingly.
            \fig{.5}{p2.jpg}

        \pagebreak
        \item \textbf{Problem 3}: \\
            In this case, with both b and the clock in the sensitivity list, but
            only updating $x$ when both are high, we need to detect any time the
            clock or b have a rising edge while the other is high. So feed the
            two signals into an AND gate and detect a rising edge on the AND
            gate's output. When either signal has a rising edge, if the other
            signal is high, the AND gate will have a rising edge.
            \fig{.4}{p3.jpg}

        \item \textbf{Problem 4}: \\
            In this problem, only the clock is in the sensitivity list. We then
            check if the clock is high before continuing. Therefore, activity
            should only be triggered on the rising edge of the clock. However,
            we then demand a rising edge on $b$. This means that this circuit is
            looking for the extremely rare condition of simultaneous rising
            edges on both the clock and on $b$. We won't be able to ensure that
            the two rising edges are exactly simultaneous, but what we do do is
            detect whether or not $b$ has experience a rising edge within the
            propagation delay time of a DFF. Begin by clocking a DFF on the
            rising edge of the clock. Feed $b$ into the DFF. Feed the
            $\overline{Q}$ output of the DFF into an AND gate with $b$. This
            is essentially ANDing $b$ with the value of $\overline{b}$. Now, if
            $b$ has a rising edge within the propagation delay of the DFF, the
            $\overline{Q}$ output will be high, and $b$ will go high, meaning
            there will be a very short pulse on the output of the AND gate.
            In only this case, we will update $x$ to be the value of $a$. This
            is a horrible circuit that essentially depends on a glitch, and I
            don't recommend anyone ever build it.
            \fig{.5}{p4.jpg}
        \pagebreak

        \item \textbf{Problem 5}: \\
            This is a simple mux. Outside of a process this is simply concurrent
            logic that udates whenever an input to the mux changes. Note that
            this will synthesize to a 4:2 mux, but there are only 3 inputs, so
            the 4th input is tied to 0.
            \fig{.5}{p5.jpg}

        \item \textbf{Problem 6}: \\
            Luckily, every input to this mux is included in the sensitivity
            list, so that the circuit does not change from the previous problem.
            When any input to the mux changes, the output will update so that we
            essentially have concurrent logic. Notice again we must tie off one
            of the inputs to the mux to 0.
            \fig{.5}{p5.jpg}
        \pagebreak

        \item \textbf{Problem 7}: \\
            Since now only the control signals are included in the sensitivity
            list, we need to essentially cache the values of the inputs to the
            mux according to when the control signals are updated. To do so, use
            3 DFFs, one for each input to the mux. Clock each of the DFFs on an
            AND gate that will go high for the rising edge of the control
            signals that correspond to that input. Then, when the control
            signals are updated, the DFF will store the value of the input at 
            that time, and pass it through accordingly. We once again have to
            tie off the 4th input to the mux to 0.
            \fig{.5}{p7.jpg}

        \item \textbf{Problem 8}: \\
            Notice in the code that (a) $x$ is a signal, and (b) instead of
            using if/elsif/else, each case is a separate if statement. As a
            result of (a), any change to $x$ is scheduled at time $t + \Delta$,
            and as a result of (b), the if statement priorities will be last to
            first, meaning if the third to last if statement is true, it will
            overwrite the value of $x$ set by the second to last or first if
            statement. Therefore, the circuit should essentially check each if
            condition. Then assign $x$ according to the first condition AND 
            the exclusion of all other conditions, OR the second condition AND
            the exclusion of all later conditions, etc. There is no default
            assignement for $x$, so if none of the conditions are met, $x$ will
            not be updated.
            \fig{.5}{p8.jpg}
        \pagebreak

        \item \textbf{Problem 9}: \\
            In this case, there are two notable changes. Firstly, we are now
            using if/elsif/else. This switches the order of precedence so that
            if the first condition is true, we will not check any later
            conditions so that earlier conditions supercede later ones. Secondly
            if none of the conditions are true $x$ will take on $e$ as a default
            now instead of not updating.
            \fig{.5}{p9.jpg}

        \item \textbf{Problem 10}: \\
            In this case, we have reverted to using if/then statements alone,
            reversing the order of priority back to the last cases taking
            precedence over earlier ones. Notice that before the if statements
            are evaluated, $x$ is set to $e$. However, remember that since $x$
            is a signal, this assignment is scheduled for time $t + \Delta$.
            This means this assignment will be overwritten by any if condition
            that is true. Therefore, this assignment to $e$ will function like
            a default if none of the if conditions are met.
            \fig{.5}{p10.jpg}
        \pagebreak

        \item \textbf{Problem 11}: \\
            This is a simple 8:3 mux. Since $i$ is an integer ranging up to 7 it
            will take 3 bits to represent all 8 values. We will simply pass each
            bit of the 8 bit opcode into a different input of the mux and set
            the control signals to the 3 bit value of $i$ so that the mux will
            output the value of the bit corresponding to $i$.
            \fig{.5}{p11.jpg}

        \item \textbf{Problem 12}: \\
            For this problem, recall that $x$ is a signal. This means that any
            changes to the value of $x$ are scheduled for time $t + \Delta$. So
            for this code, setting $x$ to $a$ is scheduled but not executed.
            This assignment is then overwritten by each successive assignment to
            $x$. Thus, no execution of the loop will have any impact on $x$
            except the final iteration. Therefore we will very simply set $x$ to
            $x$ AND opcode$[7]$. Since the opcode is in the sensitivity list,
            and changes in MSB of the opcode are the only thing that can
            possibly change the value of $x$ we don't need to worry about
            timing or clocking.
            \fig{.3}{p12.jpg}
        \pagebreak

        \item \textbf{Problem 13}: \\
            For this problem, we need to build a comparator that checks if the
            upper nibble of opcode is less than the lower nibble. Then, we use
            the output of the comparator which I named $LT$. If $LT$ is high, we
            set $x$ to $a$, otherwise we set $x$ to $x$. To simplify the logic
            of the comparator, I used the Quine-McCluskey python package I
            developed for homework 3. The full code of the package and the
            script which uses it to generate the prime implicants chart is
            attached. The prime implicants chart for the comparator is also
            attached. I then implemented the strict SOP form from the prime
            implicants chart in the given archtecture resulting in the
            following.
            \fig{.75}{p13.jpg}

        \item \textbf{Problem 14}: \\
            Use the same comparator logic from the previous problem. I
            therefore implemented the strict SOP form from the prime implicants
            chart for the comparator in the given archtecture. Note I ommitted
            several of the muxes from most of the CLBs as they were never used.
            Apologies for the handwritten labels.
            \fig{.65}{p14.jpg}

    \end{enumerate}

    \pagebreak

    \begin{scriptsize}

        \verbatiminput{../Q-M/qm.py}
        \pagebreak
        \verbatiminput{red.py}
        \pagebreak
        \begin{landscape}
                \verbatiminput{out.txt}
        \end{landscape}
    \end{scriptsize}

\end{document}
