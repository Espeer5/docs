\documentclass[fleqn]{article}

%%%%%%%%%%%%%%%%%%%% PACKAGE SETUP %%%%%%%%%%%%%%%%%%%%
% Packages without explicit setup
\usepackage{amssymb, amsmath}
\usepackage{mathtools}
\usepackage{enumitem}
\usepackage{titlesec}
\usepackage{float}
\usepackage{wrapfig}
\usepackage{multicol}
\usepackage{graphicx}
\usepackage{needspace}
\usepackage{comment}
\usepackage{subcaption}
\usepackage{bm}
\usepackage{multirow}

% Setup the geometry package (for page layout) and spacing.
\usepackage[margin=1.0in]{geometry}

\setlength{\parskip}{6pt plus 2pt minus 2pt}
\setlength{\parindent}{0in}

% Load the fancy headers package, using the defaults.
\usepackage{fancyhdr}
\pagestyle{fancy}
%\fancyfoot[C]{\thepage}  
%\renewcommand{\headrulewidth}{0pt}
%\renewcommand{\footrulewidth}{0pt}

% Setup the listings, including using color.
\usepackage{listings}
\usepackage{xcolor}
\lstset{
  showlines   = true,
  numbers     = left,
  frame       = single,
  basicstyle  = \footnotesize\ttfamily,
  basewidth   = 0.5em,
  xleftmargin = 0.33in,
  keywordstyle = \color{blue},
  commentstyle = \color{teal},
  stringstyle  = \color{violet},
}
\lstdefinestyle{continueyellow}{
  firstnumber     = last,
  backgroundcolor = \color{yellow},
  aboveskip       = -\smallskipamount,
}
\lstdefinestyle{continue}{
  firstnumber     = last,
  aboveskip       = -\smallskipamount,
}

% Additional math operators.
\DeclarePairedDelimiter\abs{\lvert}{\rvert}
\DeclarePairedDelimiter\floor{\lfloor}{\rfloor}
\DeclarePairedDelimiter\norm{\lVert}{\rVert}

\DeclareMathOperator{\asin}{asin}
\DeclareMathOperator{\acos}{acos}
\DeclareMathOperator{\atan}{atan}
\DeclareMathOperator{\atantwo}{atan2}	% Using '2' in the operator messes up.

\newcommand{\overbar}[1]{\mkern 1.5mu\overline{\mkern-1.5mu#1\mkern-1.5mu}\mkern 1.5mu}

\usepackage{textcomp, gensymb}
\renewcommand\deg{\degree}


%%%%%%%%%%%%%%%%%%%% START OF DOCUMENT %%%%%%%%%%%%%%%%%%%%
\begin{document}

% Set the header/title.
\fancyhead[L]{ME/CS/EE 133a, Fall 2024-25}
\fancyhead[R]{Git Recitation Notes}
\begin{center}\LARGE Git Recitation Notes \end{center}

\section*{Intoduction - What is Git?}
Git is what's known as a \textit{distributed version control system} or DVCS.
It was invented by Linus Torvalds, the creator of Linux, in 2005 to reake the
place of centralized version control systems (SVCS) used by Linux developers.

Version control systems are used to track changes in files and directories over
time. All VCSs have the following elements:

\begin{itemize}
    \item \textbf{Repository:} A database that stores all the changes to files
                               and directories.
    \item \textbf{Working Directory:} A directory on your computer where you can
                                      edit files from the repository.
    \item \textbf{Commit:} A snapshot of the repository at a given time.
    \item \textbf{Staging Area:} A cache of changes from the working directory
                                that are ready to be committed.
\end{itemize}

Git, and other DVCS, are distributed in that every user has a complete copy of
the repository on their computer, rather than always relying on a single
central server upstream. This allows for more flexibility and robustness in
collaborative development.

\begin{center}
    \includegraphics[scale=0.2]{cvcs.png}
    \includegraphics[scale=0.2]{dvcs.png}
\end{center}

\section*{Setting Up}

To start using Git, you need to first configure your system with your name and
email address. This information is used to identify you as the author of commits
you make.

\begin{lstlisting}
git config --global user.name <Your Name>
git config --global user.email <Your Email>
\end{lstlisting}

Now you are ready to create a new repository. \texttt{cd} to the directory where
you want to create the repository and run the following command:

\begin{lstlisting}
git init
\end{lstlisting}

This will create a \texttt{.git} folder under the current directory, which is
where Git stores all of the information about the repository.

\section*{Committing}
The most basic operation in Git is committing. This is the process of taking a
snapshot of the repository at a given time, including all of the changes that 
have been moved from the working directory to the staging area. Every commit
has the following elements:

\begin{itemize}
    \item \textbf{Author:} The person who made the commit.
    \item \textbf{Date:} The date and time the commit was made.
    \item \textbf{Message:} A short description of the changes made in the
                            commit.
    \item \textbf{Hash:} A unique identifier computed for the commit.
\end{itemize}

To add your changes to the staging area:
\begin{lstlisting}
git add <filepath>  # Add a single file
git add <directory> # Add all files under a directory
git add .           # Add all files under the working directory
\end{lstlisting}

To commit your changes to the repository:
\begin{lstlisting}
git commit -m <Your commit message here>
\end{lstlisting}

It may be useful to see the history of commits in the repository, who made them,
and what the message on each commit is. To do this:

\begin{lstlisting}
git log                        # See a detailed log of all commits
git log --oneline              # See a one-line summary of each commit
git log -n <number of commits> # See the last n commits
\end{lstlisting}

It may also be useful to see the differences between two commits, or between
the working directory and a commit. To do so:

\begin{lstlisting}
git diff                             # Compare the working directory to the staging area
git diff <commit hash>               # Compare a commit to the working directory
git diff <commit hash> <commit hash> # Compare two commits
\end{lstlisting}

Finally, you may want to reset certain files or the entire repository to a
previous commit (notice that you CAN LOSE WORK this way if you aren't careful):

\begin{lstlisting}
git reset HEAD             # Reset the staging area to the last commit
git reset HEAD <filepath>  # Reset a single file in the staging area
git reset --hard HEAD      # Reset the entire repository to the last commit
git reset --hard HEAD~<n>  # Reset the entire repository to n commits ago
git reset --hard <commit>  # Reset the entire repository to a specific commit
\end{lstlisting}

\pagebreak

\section*{Branching}

Using a DVCS like Git allows users to create branches, which are separate
lines of development split off from a commmon origin that can be worked on
independently. 

\begin{center}
    \includegraphics[scale=0.3]{branch.png}
\end{center}

The HEAD pointer in Git is a reference to the current branch you are working on.
This means that your working directory will reflect the state of the branch
that HEAD is pointing to. Placing the head pointer on a branch is called
checking out a branch.

To see the current branch and the branches that exist in the repository:

\begin{lstlisting}
git branch
\end{lstlisting}

To create a new branch:

\begin{lstlisting}
git branch <branchname>        # Create a new branch but don't check it out
git checkout -b  <branchname>  # Create a new branch and check it out
\end{lstlisting}

To switch to a different branch:

\begin{lstlisting}
git checkout <branchname> # Switch to an existing local branch
git switch   <branchname> # Switch to an existing remote branch
\end{lstlisting}

Very often, you will want to merge the changes from one branch into another. 
For example, say I branched off from the main branch and added a new feature 
to my code. I now need to merge my changes back into main. To do this:

\begin{lstlisting}
git checkout main
git merge <branchname>
\end{lstlisting}

\begin{center}
    \includegraphics[scale=0.3]{merge.png}
\end{center}

What if someone else added some changes to the main branch while I was working
on my feature? This is where merge conflicts can arise. Git will try to merge
the changes automatically, but if it can't, it will ask you to resolve the
conflict manually.

\pagebreak

\section*{Rebasing}

Rebasing is another way to integrate changes from one branch into another.
It works by moving the commits from one branch to another, and then replaying
the commits on top of the branch you are rebasing onto. For example, say you
have a feature branch that you aren't ready to merge into main yet, but you
want to keep it up to date with the main branch. You can rebase the feature
branch onto the main branch to incorporate the changes from main.

\begin{center}
    \includegraphics[scale=0.4]{rebase.png}
\end{center}

Rebasing can be risky because it rewrites the commit history of the branch you
are rebasing. This can cause problems if you are working on a branch that is
shared with other people. I recommend always backing up your branch before 
attemping a rebase. To backup your branch:

\begin{lstlisting}
git branch backup
\end{lstlisting}

To rebase your branch onto another branch:

\begin{lstlisting}
git checkout <branchname>
git rebase <branchname>
\end{lstlisting}

A powerful type of rebase is an interactive rebase. This allows you to edit,
squash, or delete commits at will as you rebase. To do this:

\begin{lstlisting}
git rebase -i <branchname>             # Rebase onto another branch
git rebase -i HEAD~<number of commits> # Rebase onto the last n commits
git rebase -i <commit hash>            # Rebase onto a specific commit
\end{lstlisting}

These commands will open a text editor with a list of commits that you can
edit. You can change the order of commits, squash commits together, or delete
commits entirely. The text editor will include instructions on how to do this.

\pagebreak

\section*{Remote Repositories}

So far, we have only been working with a local repository on our computer.
However, Git is designed to work with remote repositories as well. A remote
repository is a copy of the repository that is stored on a server, and can be
accessed by multiple users. The most common way to interact with a remote
repository is through a service like GitHub, GitLab, or Bitbucket.

To copy a remote repository to your local machine:

\begin{lstlisting}
git clone <repository URL>
\end{lstlisting}

This will automatically add the remote repository as a remote called
\texttt{origin}.

Note that you may have to set up authentication to access the remote repository.
This is done using an SSH key, and you should find instructions on how to do
this in the documentation of the service you are using.

Say instead that you have a local repository and you want to add a remote
repository to it. To do this:

\begin{lstlisting}
git remote add <remote name> <repository URL>
\end{lstlisting}

By convention, the main remote should be named \texttt{origin}. By the power of
DVCS, you can have multiple remotes for a single repository.

Remember that with a DVCS, you have a complete copy of the repository on your
local machine. This means that your local repository can have multiple branches
that are not present in the remote repository. To synchronize your local
branches with the remote repository, you set what is called an upstream branch.
This is simply a mapping between a local branch and a remote branch. Once an
upstream is set, you simply use \texttt{git push} to push changes to the remote
branch, and \texttt{git pull} to pull changes from the remote branch.

Take the following example:

\begin{lstlisting}
git clone <repository URL>    # Clone repo from remote
git checkout -b feature       # Create a new feature branch locally

(Work on feature branch)

git push -u origin feature    # Push changes to remote and set upstream

(Work on feature branch more)

git push                      # Push changes to remote (upstream already set)
\end{lstlisting}

It's best practice to push your new commits to the remote frequently as a 
backup mechanism. This way, if your computer crashes, you won't lose your work.

When you are ready to merge your branch into the main branch on the remote, you
do so by creating a \emph{pull request} or PR. This is a request to the owner
of the main branch to merge your changes into the main branch, which will be
reviewed through the PR interface on the remote repository. The PR interface
will allow you to see the changes that will be made and conduct code reviews and
discussions with other developers.

\pagebreak

\section*{Typical Workflow}

A typical workflow (called feature branching) for using Git in a collaborative
environment is as follows:

\begin{enumerate}

    \item Create a remote repository on a service like Github, whose main
          \texttt{main} branch serves as the official version of the code.
    \item Each team member clones the repository to their local machine.
    \item Each team member works on new features of the code in parallel, in
          their own branches.
    \item Each team member pushes their changes to a new branch on the remote
          set as the upstream of their branch.
    \item Each team member creates a pull request to merge their branch into
          the main branch.
    \item The branch is reviewed by other team members, who can leave comments
          and request changes.
    \item Once a team member's pull request is approved, they rebase their
          branch onto the main branch to incorporate any changes that have been
          made since they branched off, and  \emph{squash} their changes into a
          single commit (using an interactive rebase).
    \item The team member then merges their branch into the main branch.
\end{enumerate}

\section*{Conclusion}

Git is a powerful tool that can be used to manage changes to code in a
collaborative environment. It is important to understand the basic commands
and concepts of Git in order to use it effectively. The best way to learn Git
is to use it, so I encourage you to practice using best Git practices in your
upcoming final projects. There are many more advanced features of Git that I
have not covered in this recitation, so I encourage you to explore the Git
documentation and tutorials to learn more. If you get stuck or have any trouble
with Git during the projects, don't hesitate to ask me for help!

\end{document}
