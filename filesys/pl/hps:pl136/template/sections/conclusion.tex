In this paper I have used virtual digitalism to argue that the inhabitance of
virtual worlds, such as the one simulated by Nozick's Experience Machine, offer
authentic experiences with virtual objects and agents. I have shown that the 
experiences offered by virtual world technologies provide the necessary 
ingredients for living the good life, and that therefore the good life can be
lived in virtual worlds. This conclusion has important implications for the
future of virtual reality technologies and the way we think about the value of
virtual experiences. The good life is not limited to the physical world, but can
be lived in virtual worlds as well.

It is worth noting that empirical studies have shown that people tend to reject
the possibility of living in Nozick's experience machine \citep{Hindriks2018}.
This may be thought to undermine my argument by showing that people estimate
that the experiences in the EM are less valuable and would have adverse effects
on their well-being. However, it has also been showing that people are very poor
predictors of what will make them happy \citep{Gilbert2005}. In any case, 
I do not claim that people should be made to live in virtual worlds if they 
don't wish to. I claim only that the good life can be lived in virtual worlds,
establishing the basis for a new way of thinking about the value of virtual
experiences.
