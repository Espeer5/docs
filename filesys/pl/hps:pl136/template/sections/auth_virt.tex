Virtual digitalism gives us tools to analyze the authenticity of experiences
within Nozick's EM. Recall that Nozick's argument was based on the false and 
illusory nature of experiences in the EM. However, having established that the
EM is a virtual world, applying the framework of virtual digitalism shows that 
experiences in the EM are not illusory, but real experiences of a digital world.
Whether or not experiences had in this world are valuable to one's well-being is
now a different question - rather than asking whether fictional or illusory 
experiences can be valuable, we are asking whether experience of digital
objects, persons, or events can be valuable. To answer this question, I will
consider 3 pertinent dimensions of well-being drawn from the literature and
their applicability to virtual worlds: pleasure, achievement, and identity.

\subsection{Pleasure}
Pleasure is a common dimension of well-being associated with the hedonic theory
of well-being, and it provides the most straightforward case for the value of
virtual experiences. In the hedonic theory, the volume of pleasurable
experiences as compared to painful ones is the primary determinant of
well-being. Different concepts of ``pleasure'' define different hedonic theories
of well-being described by different utility functions
\citep{kahneman1999objective}, but any of them can be applied to virtual
experiences.

Early hedonic theories of well-being focused on the intensity and duration of 
physical pleasures and pains. Physical sensations are essentially a prerequisite
for the sense of presence in virtual reality, and it is clear that any virtual
environment meeting the qualifications for a virtual world will be able to 
provide physical sensations to the user. In Nozick's EM, physical sensations
are simulated for the user through direct stimulation of the user's brain and
experienced as ordinary physical sensations. Even in the limited VR systems of
today, users interact with controllers and other devices which use haptics to
provide physical sensations to the user, with capabilities continually evolving
\citep{WANG2019136}. There is no reason to believe that the physical sensations
provided by our experience in virtual worlds would be any less pleasurable,
intense, or long-lasting than those in the physical world, so that it is natural
to expect that virtual experiences can provide physical pleasure to a user in
the same way that physical experiences can.

More modern approaches to hedonism add mental and emotional dimensions to the 
utility function by which pleasure is measured. An analysis of how mental and 
emotional pleasures can be experienced in virtual worlds reveals that they are
also well-suited to provide these types of pleasure. Mental pleasures are 
pleasures derived from intellectual activities, such as solving puzzles or
learning new information. Virtual worlds can clearly provide these pleasures in
equal measure to the physical world. For example, the experience of solving a 
jigsaw puzzle in a virtual world could be simulated to be identical to the
experience of solving a physical jigsaw puzzle, and will therefore provide the
same mental pleasure. Virtual worlds may also unlock doors to mental pleasures
that are not possible in the physical world, such as the pleasure of exploring
a virtual world that is physically impossible to create in the physical world,
solving a puzzle that cannot be realized in physical space, or communicating 
with a virtual person who is physically impossible to meet.

Emotional pleasures provide an arguably more complex case for the value of
virtual experiences, and depend in part on whether or not a user is aware they
are in a virtual world. In a case like the EM, where the user is unaware that
they are in a virtual world, emotional pleasures experienced in the virtual
world will be equivalent to those experienced in the physical world. Since the
user believes that they are experiencing real events, the emotional responses
they have to those events will be natural emotional expressions. In a case like
modern VR headsets, the user is aware they are interacting with a virtual world.
Users will be much less likely to experience intense emotional responses to the
virtual persons they encounter, as they will recognize the ontological
limitations of agents within the virtual environment. However, they could still
experience strong emotional responses to other users of the virtual world in
the same way that they would in the physical world. For example, a user could
fall in love with another user in a virtual world, analogous to how individuals
may form emotional bonds through online platforms. The emotional response is
real, and obtains through the inhabitance of the virtual world, demonstrating
that virtual worlds can provide rich emotional pleasures to users.

\subsection{Achievement}
Achievements and their role in well-being are a central focus of the eudaemonic
theory of well-being. Proponents of \emph{achievementism} argue that
achievements serve as a mechanism to carve out and obtain those desires that are
relevant to one's well-being. Through the pursuit of achievements, individuals
promote their own well-being by satisfying those desires relevant to the
actualization of their potential \citep{Bradford2016-BRAAWA-3}.

One might argue that the events or objects in virtual worlds are fundamentally
different from those in the physical world and these differences make it
impossible for them to be achievements. However, this perspective is dismissed
with a simple thought experiment. Imagine an advanced computer scientist 
implementing a highly sophisticated and theoretical algorithm. The scientist
works for years on the algorithm, and finally, after much effort, they are able
to implement the the algorithm in a programming language and run it to generate
an output. This is something that is clearly an achievement by any measure. This
begs the question: what precisely was achieved? An algorithm was implemented and
an output was generated. Both of these can be summed up as the manipulation of
bits in a computer. The same final organization of bits could have been obtained
much more simply by hardcoding the output, or typing ones and zeros into the
computer, yet we know that the implementation of the algorithm is a much greater
achievement because of the symbolic meaning of the bits. Consider, by contrast,
a user engaging with a VR environment, who beats a game and obtains a virtual
trophy. ``Beating the game'' and ``a virtual trophy'' are both simply
manipulations of bits in a computer system, made valuable by their symbolic
meaning. The only difference is how the manipulations were brought about. The
user in VR had to interact with the virtual world in particular ways whereas the
scientist had to interact with the physical world in particular ways, but the
outcome is functionally equivalent in terms of symbolic structure and effort.
Therefore is is clear that the question is not whether virtual achievements are
possible, but rather the conditions of interaction with a virtual world under
which achievements can be meaningful.

In this paper, I will depend on Laurence James' definition of
\emph{m-achievements} as those achievements that are meaningful and increase
one's well-being \citep{James2005}. If virtual worlds can provide m-achievements,
then they can provide a valuable source of achievement-based well-being. An
achievement is an m-achievement if and only if, for individual $x$,
\begin{enumerate}
    \item $x$ would have a cause to reassess themselves if they did not achieve
          it.
    \item When $x$ achieves it, $x$ can justifiably increase their
          self-conception.
    \item It is hard for the average person to do.
    \item It is hard for $x$ to do.
\end{enumerate}
We will work through these conditions in reverse order.

The third and fourth conditions are straightforward. Solving difficult puzzles
in the EM or in modern VR headsets is clearly just as hard as solving them in 
the physical world. Note that this doesn't imply \emph{equal} difficulty between
a physical achievement and its virtual counterpart - for example, climbing
virtual Mount Everest is likely easier and less dangerous than climbing the 
physical Mount Everest - but it doesn't need to be equal. The important point is
that the virtual achievement is hard for the individual to do. Some virtual 
achievements could be harder than physical ones - for example, a virtual
achievement that requires the user to solve a puzzle that is physically
impossible to create could be harder than any physical puzzle.

The second condition is also straightforward. One is able to increase their self
conception when they successfully complete an action which they were previously 
unsure was within their capabilities. This is clearly true in virtual worlds. 
Just because the domain in which I demonstrate the new capability is virtual
doesn't mean that the capability is any less real. However, this point requires
careful analysis through the lens of virtual digitalism. Perhaps I am not sure
in the real world if I have the capability of dunking a basketball. However, if 
I enter a virtual world where my avatar is 8 feet tall and dunk the basketball,
I haven't demonstrated that I can dunk a basketball, I've demonstrated that I
can dunk a \emph{virtual} basketball. This case would not allow me to advance my 
self-conception, as if you had previously asked me if I could dunk a virtual 
basketball in VR where I was 8 feet tall, I would have clearly said yes. Of
course, if on the other hand, I was unsure if I could dunk a virtual basketball
under the conditions of a particular virtual world and then did so, I would have
increased my self-conception. 

The first condition is similarly clear. If I set out to achieve a virtual
achievement, investing many hours of virtual world inhabitance and much effort 
towards it, and fail to achieve it, I would have cause to reassess myself. There
is little to no difference between this condition in the physical world and in
the virtual world. Just because all of my time and effort into the task was
spent interacting with virtual objects in a virtual world in no way diminishes 
the value of that time or effort, and therefore the value of the achievement.

\subsection{Identity}
Identity appears to be a complex but critical dimension of well-being. Our
psychological capability to rationalize our experiences and actions in terms of
our self-conception is a key component of our well-being \citep{McAdams2013}.
Actions and events that are consistent with our self-conception are likely to
increase our well-being, while those that are inconsistent are likely to be
aversive. Virtual worlds add a new and highly uncertain dimension to this area,
as they allow people to take on new identities and embody different physical 
forms across different virtual worlds \citep{Lin2022}. Research in this area is
still in its infancy, but I would argue that whether or not inhabitance of
virtual worlds can \emph{always} provide a positive contribution to one's
well-being through identity, it can certainly do so in many cases.

Consider the case of the EM. In the EM, the user acts fully as themselves. Since
they are unaware they are inhabiting a virtual world, every event and action
will have precisely the same impact on their identity as it would in the
physical world. We can use this as a basis to consider what happens in other 
virtual worlds where the user may be aware they are inhabiting a virtual world. 
Perhaps in said world, the user acts as themselves fully, and creates an avatar
which is an accurate representation of themselves. In this case, I think it is
clear that the user's actions and experiences in the virtual world will have a 
\emph{very similar} impact on their identity as they would in the physical
world. The user still acts as themselves, and attributes their actions,
abilities, and thoughts to themselves.

Now consider a virtual world in which the user plays as a character who is not
themself, and the user is aware they are inhabiting a virtual world. In this
case, the user's choice to embody a different character will have an impact on
the self, however this impact need not be negative. Perhaps the user is playing
the part of a character they would like to be, say, more confident or decisive.
Or the user could play as a character with whom they identify more strongly than
with their physical self - for example, a user who identifies as a different
gender could easily embody their gender in a virtual world. These examples
suffice to show that inhabitance of virtual worlds can provide a positive
contribution to one's well-being through identity, even if there exist cases in
which virtual identity embodiment may not contribute positively.