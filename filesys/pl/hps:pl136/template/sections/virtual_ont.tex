Key to analyzing the ``authenticity'' of experiences in virtual worlds is
developing a clear ontology accounting for the nature of virtual objects and the
relationships between virtual objects and the physical world. David Chalmers
provides a useful framework for understanding virtual objects which he refers 
to as ``virtual digitalism,'' with 4 key aspects \citep{Chalmers2017-CHATVA-3}:
\begin{itemize}
    \item Virtual objects really exist and are digital objects.
    \item Events in digital worlds are largely digital events that really take
          place.
    \item Experiences in virtual worlds involve non-illusory perceptions of a
          digital world.
    \item Virtual experiences of a digital world can be about as valuable as
          experiences of the physical world.
\end{itemize}
This perspective is in contrast to the dominant view in classical philosophy
called ``virtual fictionalism'' which holds that events and objects in virtual
worlds are not real, but merely fictional entities. 

Virtual digitalism can be described as the \emph{it from bit} ontology of
virtual worlds, holding that virtual objects are real objects whose existence is
grounded in the digital information that constitutes them
\citep{Chalmers2022-CHARVW}. So a virtual rock is a real rock, but it is a rock
that exists in a digital world and is made of bits rather than in the physical
world and made of atoms. An avatar embodied by a user in a virtual world is a
real body, but a real virtual body made of bits.

Importantly, this does not mean that virtual objects are identical to physical
objects that they may resemble. For example, if I keep a dog as a pet in a 
virtual world, I don't have a real physical dog, but I do have a real virtual
dog. The meaning of ``dog'' is different in the virtual world than in the
physical world, and my virtual dog will have many characteristics that a
physical dog would not share, yet it makes the dog no less real. My pet is made
of bits and may have programmed behaviors, but I have a pet nonetheless.

Similarly, interacting with a virtual facsimile of a person or object that
exists in the virtual world is not the same as interacting with the person or
object in the physical world, but consitutes a real interaction with a real 
(virtual) person or object. To borrow an example from Chalmers, if I have a 
conversation with virtual Barack Obama in virtual reality, I am certainly not
conversing with the Barack Obama from my physical reality, but I am conversing 
with a different virtual person who is real in the virtual world. Virtual people
may vary widely from their physical counterparts - depending on the world, they
may be more or less intelligent, lack memories, lack consciousness, etc. - and
these factors may certainly affect the value of the interaction. Conversing with
virtual Obama is likely a lower quality conversation and less valuable to me
than conversing with physical Obama, but this value judgement is not based on
some ``illusive'' or ``fictional'' aspect of the virtual Obama, merely on the
differences between the two real Obamas.
