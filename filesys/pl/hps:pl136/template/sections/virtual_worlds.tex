A variety of attempts have been made to define virtual worlds throughout the
philosophical literature. It is difficult to adopt a definition sufficient to 
provide a demarcation between virtual worlds and other types of worlds.
Consider the case of augmented reality (AR) technologies, which overlay virtual
objects onto the physical world. Does this constitute a virtual world? Some
combination of virtual and physical worlds? Or is it simply a new way of
interacting with the physical world? The difficulty of defining virtual worlds
is compounded by the fact that the technologies that enable them are rapidly
evolving, and the experiences they provide are becoming increasingly
sophisticated.

For the purposes of this paper, I will use a definition proposed by Carina
Girvan, who defines a virtual world as a simulated environment which meets the
following criteria for a world \citep{Girvan2018}:
\begin{enumerate}
    \item It is a spacial structure inhabited and shaped by agents.
    \item Experiences and their interpretations are mediated by physical and
          psychological responses of the inhabitants.
    \item Inhabitants use their bodies to move around the world and interact
          with the environment and other inhabitants.
\end{enumerate}
I will clarify the second and third criteria here by stating that the body of 
an inhabitant is not limited to the body in physical reality, but also includes
the embodiment of the inhabitant in the virtual world. For example, if a VR 
technology succeeds in facilitating the presence effect such that I feel as
though I am embodying the avatar of a bird flying through the sky, then my 
physical responses and interaction with that virtual body are sufficient to 
meet these criteria.

This definition of virtual worlds is broad enough to encompass a wide variety
of technology, while excluding those that do not provide the sense of presence
necessary for a world. Modern VR headsets are clearly included - they provide a
fully simulated spatial structure in which users move around and interact with
the environment and other users. Their experience is shaped by their physical
and psychological responses, and they can shape the world in turn. AR
technologies also fit this definition - though they rely on the physical world 
for portions of their spatial structure, they provide a simulated environment
that users can interact with and shape. This definition successfully excludes
technologies which do not facilitate the sense of presence necessary for a
world, such as a movie or even a first-person video game. Though a movie may
provide a simulated environment, it does not meet the criteria of inhabitation
and shaping by agents, as the viewer is a passive observer. A first-person video
game may meet the first and second criteria, but importantly, I do not have the
embodied experience of my avatar in the game in the way I would expect in a 
virtual world.

Nozick's EM can be understood as a virtual world under this definition. The EM
is a simulated environment providing a spatial structure in which the user can
move around and interact with the environment. Through neural stimulation, the
user is provided an embodied experience in the EM, and their physical and
psychological responses shape their experience. The EM is quite an extreme case
of a virtual world, as it maintains a higher level of isolation of the user from
the physical world than most modern VR technologies. Given this, if I can
demonstrate that the EM can provide authentic experiences, I will have shown
that the virtual worlds accessible through modern and upcoming VR technologies
can also provide authentic experiences.
