Virtual and augmented reality (VR/AR) technologies are rapidly becoming more
sophisticated, offering users increasingly offer users the experience of
simulated presence. As these technologies advance, they raise important
questions about the nature of well-being in virtual worlds. Classical accounts
emphasize the importance of authenticity for well-being, and deny that virtual
worlds can provide authentic experiences. This paper combines insights from
the philosophy and psychology of well-being to argue that virtual worlds
can provide authentic experiences, and analyzes the quality of experiences in
virtual worlds over several dimensions relevant to well-being to show that one
can live \emph{The Good Life+}: The Good Life in a virtual world.
