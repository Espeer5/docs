In 1971, Robert Nozick introduced a thought experiment known as the Experience
Machine (EM). The EM is a hypothetical device that can simulate any experience
the user desires, such that they cannot distinguish between the simulated
experience and reality. Nozick asks us to consider whether we would choose to
plug into the EM, and argues that most people would choose not to. Nozick says 
of such a choice, ``plugging into an experience machine limits us to a man-made
reality, to a world no deeper or more important than that which people can
construct. There is no actual contact with any deeper reality, though the
experience of it can be simulated'' \citep{Nozick1974-NOZASA}. Nozick's 
perspective can be summed up as a two part claim:
\begin{enumerate}
    \item Authenticity is a necessary condition to live a good life.
    \item Life in the EM is necessarily inauthentic.
\end{enumerate}

Nozick's argument was developed in the context of a debate between two schools 
of thought in the philosophy of well-being: hedonism and eudaemonism. Hedonists
hold that pleasure is the only intrinsic good, and that the good life is one
filled with pleasure. Eudaemonists, on the other hand, hold that living the good
life requires the actualization of one's human potential \citep{Ryan_Deci_2001}.
Nozick took the eudaemonic perspective, arguing that while life in the EM would
be constantly pleasurable, it would not be a good life as the inauthenticity of
life lived there would occlude the possibility of actualizing one's potential in
a meaningful way.

Authenticity here can be understood as the degree to which an experience is
genuine, or the extent to which it reflects the true nature of the world. In
this context, Nozick is arguing that the experiences in the EM are inauthentic
because they are not grounded in the real world, and therefore cannot be
meaningful. Nozick's argument has been influential in the debate over the
relationship between authenticity and well-being, and has been used to argue
against the possibility of living a good life in a virtual world
\citep{Slater2020}.
