The philosophical quest to understand happiness has been a central one in the 
history of both philosophy and psychology across the globe. Philosophers in
the Western tradition from Aristotle to Mill have sought to understand what
constitutes a good life, and how one can achieve it. In more recent years, the
fields of psychology and neuroscience have joined the discourse, providing
empirical insights into the nature of human well-being \citep{Stoll2014}. 
A crucial dimension of the debate in this field is on the topic of
\emph{authenticity}. Many traditional accounts of well-being require that
lifestyles be authentic in order to be good \citep{sep-authenticity}. As with 
many areas of philosophy, the rise of new technologies has brought new 
challenges to the debate. In particular, the rise of virtual environments in
recent years has raised significant questions about the meaning of authenticity
and its role in well-being.

Virtual and augmented reality (VR/AR) technologies are rapidly becoming more
sophisticated, offering users increasingly immersive experiences that simulate
complete virtual environments \citep{Wang_Siau_2024}. These technologies seek
(with increasing success) to give users a sense of \emph{presence} in a virtual
world, such that they feel as though they are actually there
\citep{slater2018immersion}. Already this technology has found applications in
diverse fields ranging from entertainment to education to therapy, with further
applications on the horizon \citep{ijerph191811278}. As people spend more time
inhabiting virtual worlds, it becomes increasingly important to understand the
impact of these worlds on human well-being. This paper will explore the
authenticity condition for well-being as it applies to virtual worlds, and argue
that \emph{contra} classical accounts, virtual worlds can provide authentic
experiences and contribute positively to well-being. I will begin by presenting
the traditional objection to virtual worlds as inauthentic through the lens of
Nozick's experience machine thought experiment. In sections 3 and 4 I will then
develop an explicit account of virtual worlds and their ontologies, and in
section 5 I will use this account to analyze how the inhabitance of virtual
worlds impacts well-being along several important dimensions identified from the
psychology and philosophy of well-being. I will use this analysis to argue that
one can live \emph{The Good Life+}: The Good Life in a virtual world.