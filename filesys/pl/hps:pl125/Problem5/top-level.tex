\documentclass[11pt, a4paper]{article}
\usepackage[a4paper, left=30mm, right=30mm]{geometry} % margin=2.6cm

\usepackage[utf8]{inputenc}  % allow utf-8 input
\usepackage[T1]{fontenc}     % use 8-bit T1 fonts
\usepackage{ebgaramond}

\usepackage{hyperref}         % hyperlinks
\usepackage{url}              % simple URL typesetting
\usepackage{booktabs}         % professional-quality tables
\usepackage{amsfonts}         % blackboard math symbols
\usepackage{nicefrac}         % compact symbols for 1/2, etc.
\usepackage{microtype}        % microtypography
\usepackage{xcolor}           % colors
\usepackage{bbm}
\usepackage{amsthm}
\usepackage{rotating}
\usepackage{braket}
\usepackage{physics}

\hypersetup{                 % setup the hyperref package
    colorlinks=true,
    linkcolor=blue,
    filecolor=blue,
    urlcolor=blue,
    citecolor=blue,
    bookmarks=true,
}

\usepackage{graphicx} % Required for inserting images
\usepackage[ngerman, english]{babel}
\usepackage[iso, ngerman]{isodate}

\usepackage{bbold}
\usepackage{mathtools}
\usepackage{amsmath} % Required for \DeclareMathOperator
\usepackage{nicefrac}
\usepackage{tikz}
\usepackage{subcaption}
\usepackage{centernot} % for the comparison

%\theoremstyle{definition}
\newtheorem{definition}{Definition}
\newtheorem{theorem}{Theorem}

\usepackage{verbatim}

\RequirePackage[T1]{fontenc} 
\RequirePackage[tt=false, type1=true]{libertine} 
\RequirePackage[varqu]{zi4} 
\RequirePackage[libertine]{newtxmath}

\usepackage[round,comma]{natbib}
\bibliographystyle{plainnat}

\newcommand{\monthyeardate}{\ifcase \month \or January\or February\or March\or %
April\or May \or June\or July\or August\or September\or October\or November\or %
December\fi, \number \year} 

\title{HPS/Pl 125: Problem 5}
\author{%
  Edward Speer
  \\
  California Institute of Technology\\
  HPS/Pl 125, WI '25 \\
}
\date{\monthyeardate}

\begin{document}

\maketitle

\noindent \emph{Consider the following three energy eigenstate wave functions
for an electron in a hydrogen atom:
\[\psi_{100}(r, \theta, \phi, t) = \frac{1}{\sqrt{a^3\pi}}e^{\frac{-r}{a}-i\frac{E_1t}{\hbar}}\]
\[\psi_{211}(r, \theta, \phi, t) = \frac{-1}{8\sqrt{a^5\pi}}r\sin\theta e^{\frac{-r}{2a}+i\phi-i\frac{E_1t}{4\hbar}}\]
\[\psi_{21-1}(r, \theta, \phi, t) = \frac{1}{8\sqrt{a^5\pi}}r\sin\theta e^{\frac{-r}{2a}-i\phi-i\frac{E_1t}{4\hbar}}\]}

\section{Part a} \emph{Calculate the Bohmian velocity of the electron for each
of the three wavefunctions using the standard Bohmian guidance equation}
\\ \hfill \\
\[\overrightarrow{v}_{100} = \frac{\hbar}{m}\Im[\frac{\overrightarrow{\nabla}\psi_{100}}{\psi_{100}}] = \frac{\hbar}{m}\Im[\frac{\hat{r}\frac{\partial \psi_{100}}{\partial r} + \frac{1}{r}\hat{\theta}\frac{\partial \psi_{100}}{\partial\theta} + \frac{1}{r\sin\theta}\hat{\phi}\frac{\partial \psi_{100}}{\partial \phi}}{\psi_{100}}] = \frac{\hbar}{m}\Im[\frac{\frac{-1}{\sqrt{a^5\pi}}\hat{r}\psi_{100}}{\psi_{100}}] = \boxed{0}\]
\\ \hfill \\
\[\overrightarrow{v}_{211} = \frac{\hbar}{m}\Im[\frac{\overrightarrow{\nabla}\psi_{211}}{\psi_{211}}] = \frac{\hbar}{m}\Im[\frac{\hat{r}\frac{\partial \psi_{211}}{\partial r} + \frac{1}{r}\hat{\theta}\frac{\partial \psi_{211}}{\partial\theta} + \frac{1}{r\sin\theta}\hat{\phi}\frac{\partial \psi_{211}}{\partial \phi}}{\psi_{211}}]\]
\[= \frac{\hbar}{m}\Im[\frac{(1 - \frac{r}{2a})\hat{r}\psi_{211}+\cot\theta\frac{1}{r}\hat{\phi}\psi_{211} + \frac{i}{r\sin\theta}\hat{\phi}\psi_{211}}{\psi_{211}}] = \boxed{\frac{\hbar}{2mr\sin\theta}\hat{\phi}}\]
\\ \hfill \\
\[\overrightarrow{v}_{21-1} = \frac{\hbar}{m}\Im[\frac{\overrightarrow{\nabla}\psi_{21-1}}{\psi_{21-1}}] = \frac{\hbar}{m}\Im[\frac{\hat{r}\frac{\partial \psi_{21-1}}{\partial r} + \frac{1}{r}\hat{\theta}\frac{\partial \psi_{21-1}}{\partial\theta} + \frac{1}{r\sin\theta}\hat{\phi}\frac{\partial \psi_{21-1}}{\partial \phi}}{\psi_{21-1}}]\]
\[= \frac{\hbar}{m}\Im[\frac{(1 - \frac{r}{2a})\hat{r}\psi_{21-1}+\cot\theta\frac{1}{r}\hat{\phi}\psi_{21-1} - \frac{i}{r\sin\theta}\hat{\phi}\psi_{21-1}}{\psi_{21-1}}] = \boxed{\frac{-\hbar}{2mr\sin\theta}\hat{\phi}}\]

\section{Part b} \emph{Explain in words the trajectories of the electron for
each wave function and how they compare to what one would expect from the Bohr
model of the atom.}

The trajectory of the first wave function, $\psi_{100}$ is quite unexpected. You
have a stationary electron which is not moving, simply inhabiting one point in 
space around the nucleus. This obviously contrast with the classical Bohr mdoel
of the atom in which we expect to see the electron orbiting the nucleus in a
circle.

The trajectories of the second and third wave functions, $\psi_{211}$ and
$\psi_{21-1}$, are more in line with what we would expect from the Bohr model,
but still significantly differentl. We see a velocity in the azimuthal direction,
which would correspond to the electron orbiting the nucleus. However, we see
that the velocity is inversely proportional to both the radius of the electron's
orbit and the sine of the polar angle. This means that the electron is moving
faster when it is closer to the nucleus and when it is closer to the poles of
the atom. This is in stark contrast to the Bohr model, where the electron moves
at a constant velocity in a circular orbit around the nucleus.


\end{document}
