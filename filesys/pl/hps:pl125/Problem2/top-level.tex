\documentclass[11pt, a4paper]{article}
\usepackage[a4paper, left=30mm, right=30mm]{geometry} % margin=2.6cm

\usepackage[utf8]{inputenc}  % allow utf-8 input
\usepackage[T1]{fontenc}     % use 8-bit T1 fonts
\usepackage{ebgaramond}

\usepackage{hyperref}         % hyperlinks
\usepackage{url}              % simple URL typesetting
\usepackage{booktabs}         % professional-quality tables
\usepackage{amsfonts}         % blackboard math symbols
\usepackage{nicefrac}         % compact symbols for 1/2, etc.
\usepackage{microtype}        % microtypography
\usepackage{xcolor}           % colors
\usepackage{bbm}
\usepackage{amsthm}
\usepackage{rotating}
\usepackage{braket}

\hypersetup{                 % setup the hyperref package
    colorlinks=true,
    linkcolor=blue,
    filecolor=blue,
    urlcolor=blue,
    citecolor=blue,
    bookmarks=true,
}

\usepackage{graphicx} % Required for inserting images
\usepackage[ngerman, english]{babel}
\usepackage[iso, ngerman]{isodate}

\usepackage{bbold}
\usepackage{mathtools}
\usepackage{amsmath} % Required for \DeclareMathOperator
\usepackage{nicefrac}
\usepackage{tikz}
\usepackage{subcaption}
\usepackage{centernot} % for the comparison

%\theoremstyle{definition}
\newtheorem{definition}{Definition}
\newtheorem{theorem}{Theorem}

\usepackage{verbatim}

\RequirePackage[T1]{fontenc} 
\RequirePackage[tt=false, type1=true]{libertine} 
\RequirePackage[varqu]{zi4} 
\RequirePackage[libertine]{newtxmath}

\usepackage[round,comma]{natbib}
\bibliographystyle{plainnat}

\newcommand{\monthyeardate}{\ifcase \month \or January\or February\or March\or %
April\or May \or June\or July\or August\or September\or October\or November\or %
December\fi, \number \year} 

\title{HPS/Pl 125: Problem 2}
\author{%
  Edward Speer
  \\
  California Institute of Technology\\
  HPS/Pl 125, WI '25 \\
}
\date{\monthyeardate}

\begin{document}

\maketitle

\noindent \emph{In Maudlin's figure 14, the wave function shown is of the form
                \[\psi(x)=Ae^{i\alpha x}\]
                where $A$ and $\alpha$ are real constants, and the wave function
                is normalized.}

\section{Part a} \emph{Show that $\psi(x)$ is an eigenstate of the momentum
                       operator}
\\ \hfill \\
$\psi(x)$ is an eigenstate of the momentum operator iff
$\hat{p}\psi(x)=k\psi(x)$ for some constant $k$, where $k$ is the eigenvalue of
the momentum operator, $\hat{p} = -i\hbar\frac{\partial}{\partial x}$
\[\hat{p}\psi(x) = -i\hbar\frac{\partial}{\partial x}[Ae^{i\alpha x}]
  = -i\hbar(i\alpha Ae^{i\alpha x}) = \hbar\alpha Ae^{i\alpha x} =
  \hbar\alpha\psi(x)\]

\noindent Therefore, $\psi{x}$ is an eigenstate of the momentum operator with
eigenvalue $k=\hbar\alpha$.

\section{Part b} \emph{Use the time-dependent schrödinger equation to determine
                       the way this state evolves over time, $\psi(x, t)$}
\\ \hfill \\
The time-dependent Schrödinger equation (with $V = 0$) is given by
\[i\hbar\frac{\partial}{\partial t}\psi(x, t) =
-\frac{\hbar^2}{2m}\frac{\partial^2}{\partial x^2}[\psi(x, t)]\]
Using seperation of variables, write $\psi(x, t)$ as the product of a spatial
component and a temporal component, $\psi(x, t) = \psi(x)\phi(t)$ — note that 
$\psi(x)$ is the wave function given in the problem statement. Substituting:
\[i\hbar\frac{\partial}{\partial t}[\psi(x)\phi(t)] =
-\frac{\hbar^2}{2m}\frac{\partial^2}{\partial x^2}[\psi(x)\phi(t)]\]
\[i\hbar\psi(x)\frac{\partial}{\partial t}\phi(t) =
-\frac{\hbar^2}{2m}\phi(t)\frac{\partial^2}{\partial x^2}\psi(x)\]
Seperate all time and spatial components to opposite sides of the equation:
\[\frac{i\hbar}{\phi(t)}\frac{\partial}{\partial t}\phi(t) =
-\frac{\hbar^2}{2m\psi(x)}\frac{\partial^2}{\partial x^2}\psi(x)\]
We know the derivative of $\psi(x)$ is $i\alpha\psi(x)$, so:
\[\frac{i\hbar}{\phi(t)}\frac{\partial}{\partial t}\phi(t) =
-\frac{\hbar^2}{2m\psi(x)}(-\alpha^2\psi(x)) = 
\frac{\hbar^2\alpha^2}{2m}\]
\[\frac{1}{\phi(t)}\frac{\partial}{\partial t}\phi(t) =
\frac{-i\hbar\alpha^2}{2m}\]
\[\ln(\phi(t)) = \frac{-i\hbar\alpha^2}{2m}t + C\]
\[\phi(t)=e^C e^{\frac{-i\hbar\alpha^2}{2m}t}\]
Recombine the spatial and temporal components and absorb all constants
into one normalizing constant to get the full wave function:
\[\psi(x, t) = Ce^{i\alpha x}e^{\frac{-i\hbar\alpha^2}{2m}t}\]

\section{Part c} \emph{Describe what figure 14 would look like if you animated 
                       it to show the time evolution of this wave function}
\\ \hfill \\
As I see it, there are two options for this animation depending on how you
conceive of the lines. The first option would be as follows: the lines are
placed on the regions of the wavefunction with equal phase, and animated to 
follow these lines of equal phase, such that the lines are always associated
with the same phase throughout the duration of the animation. If this is the
case, then the lines should ``move along with the wavefunction'', meaning that 
they will slide to the right at a constant speed off the page. The second option
is that the lines are drawn on regions of space with equal phase, but not
affixed to the wavefunction, such that they are stationary in space. In this
case, we should see periodic behavior of the labeled phase of each line. The 
phase at each line will oscillate between $0$ and $2\pi$ as the wavefunction
evolves over time. At each line, we should see oscillation at the same rate, and
each time one of the labels returns to its original phase, so too should all the
others.

\end{document}