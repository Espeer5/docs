\documentclass[11pt, a4paper]{article}
\usepackage[a4paper, left=30mm, right=30mm]{geometry} % margin=2.6cm

\usepackage[utf8]{inputenc}  % allow utf-8 input
\usepackage[T1]{fontenc}     % use 8-bit T1 fonts
\usepackage{ebgaramond}

\usepackage{hyperref}         % hyperlinks
\usepackage{url}              % simple URL typesetting
\usepackage{booktabs}         % professional-quality tables
\usepackage{amsfonts}         % blackboard math symbols
\usepackage{nicefrac}         % compact symbols for 1/2, etc.
\usepackage{microtype}        % microtypography
\usepackage{xcolor}           % colors
\usepackage{bbm}
\usepackage{amsthm}
\usepackage{rotating}

\hypersetup{                 % setup the hyperref package
    colorlinks=true,
    linkcolor=blue,
    filecolor=blue,
    urlcolor=blue,
    citecolor=blue,
    pdftitle={Entitlement Justice and Measures of Algorithmic Fairness},
    pdfauthor={Edward Speer, Llama-2-7b},
    pdfkeywords={entitlement justice, algorithmic fairness, AI ethics},
    bookmarks=true,
}


\usepackage{graphicx} % Required for inserting images
\usepackage[ngerman, english]{babel}
\usepackage[iso, ngerman]{isodate}


\usepackage{bbold}
\usepackage{mathtools}
\usepackage{amsmath} % Required for \DeclareMathOperator
\usepackage{nicefrac}
\usepackage{tikz}
\usepackage{subcaption}
\usepackage{centernot} % for the comparison

%\theoremstyle{definition}
\newtheorem{definition}{Definition}
\newtheorem{theorem}{Theorem}

\usepackage{verbatim}

\RequirePackage[T1]{fontenc} 
\RequirePackage[tt=false, type1=true]{libertine} 
\RequirePackage[varqu]{zi4} 
\RequirePackage[libertine]{newtxmath}

\usepackage[round,comma]{natbib}
\bibliographystyle{plainnat}

\newcommand{\monthyeardate}{\ifcase \month \or January\or February\or March\or %
April\or May \or June\or July\or August\or September\or October\or November\or %
December\fi, \number \year} 

\title{``Ethereal Non-Locality'' and Causal Influence in Quantum Mechanics}
\author{%
  Edward Speer
  \\
  California Institute of Technology\\
  HPS/Pl 125, WI '25 \\
}
\date{\monthyeardate}

\begin{document}

\maketitle

\section{Introduction}
In 1935, Einstein, Podolsky, and Rosen (EPR) published a paper,~\cite{EPR_1935},
which set up the first half of a conundrum in quantum mechanics. EPR argued
through a thought experiment that assuming locality in quantum mechanics meant 
that the $\psi$-function could not provide a complete description of a physical
system — that there must be hidden variables that pre-determine the outcomes of
measurements. Thirty years later~\cite{Bell_1964} showed there could be no such
hidden variables. Together, these two papers can be thought of as setting up the
following argument: if the predictions of quantum mechanics are correct, then
quantum mechanics is an inherently non-local theory. 

What is troubling about non-locality? Einstein was famously opposed to it,
naming the implied non-locality of quantum mechanics ``spooky action at a
distance.'' This ``action at a distance'' is the idea that two particles can be
space-like separated, and yet still be correlated in a way that is not
deterministic. This notion directly conflicts with a natural understanding of
causality, inspiring the resistance against it. Yet perhaps not all is lost. 
A general attitude towards this non-locality is shown by this quote
from~\cite{Griffiths_2020}: ``Causal influences cannot propagate faster than
light, but there is no compelling reason why ethereal ones should not. The 
influences associated with the collapse of the wave function are of the latter
type, and the fact that they 'travel' faster than light may be surprising, but
it is not, after all, catastrophic.''

In this essay, I will explore the concept of ``ethereal non-locality'' and its
implications for causal influence in quantum mechanics. I will challenge 
Griffiths' claim that ethereal influences are not catastrophic (catastrophic to
whose or what's ends?) and argue that the non-locality of quantum mechanics does 
indeed pose a major challenge to either our understanding of physics or our
philosophy of causality.

I will first present the EPR argument and Bell's theorem, and demonstrate the 
relevant technical details of quantum theory. I will then discuss the dominant
theory of causation and approach to causal inference, and show how they are
challenged by the non-locality of quantum mechanics. Finally, I will argue that
rather than accepting ethereal non-locality as a benign feature of quantum
mechanics and arguing that it is not catastrophic, we should instead take it as
a serious challenge to be addressed by future research in the philosophy of
physics, with consequences not only for our understanding of quantum mechanics,
but also for our understanding and broader application of causal inference.

\section{The EPR-Bell Argument}

The EPR and Bell arguments both deal with the concept of entanglement. Entangled
particles are those which have been prepared such that they are described by a
wave function that is not separable into the wave functions of the individual
particles, resulting in a correlation between the two particles. For simplicity,
we will sum up both EPR and Bell using the same example of an entangled system,
the GHZ state given in~\cite{GHZ_2007}. The GHZ is a three-particle entangled
state, with the following properties:
\begin{itemize}
    \item If spin is measured along the $x$-axis for all three particles, the 
          number of particles with spin up will always be odd.
    \item If spin is measured along the $x$ axis for the one particle, and along
          the $z$-axis for the third particle, the number of particles with spin
          up will always be even.
\end{itemize}

On this experimental setup, the EPR argument would ask us to consider the
following: In the case of either measurement, all $x$ spin or one $x$ and two
$z$ spins, if we measure the first two outcomes, we can predict the third
outcome with certainty. \textit{Before I ever measure} the third particle, I
know what the outcome of the measurement will be. However, I could not predict
this outcome from the preparation of the state alone. This means that one of two
things happened: either the state of the third particle was determined at the
time of preparation, or the measurement of the first two particles somehow
influenced the outcome of the third particle instantaneously. EPR assume
locality to hold, and so they conclude that the state of the third particle must be
determined prior to measurement. However, the key takeaway is that if quantum
mechanics is local, then the $\psi$-function cannot provide a complete
description of the physical system — there must be some hidden variable which 
determined the outcome prior to the measurement which was not captured in the 
$\psi$-function.

Bell's work, on the other hand, shows that there can be no such hidden variable.
If the outcomes of the measurement are determined by some hidden variable prior
to measurement, then there must be some way to assign outcomes of measurements 
to each of the particles in the state before measurement. Bell showed that this
cannot be done — There is no way to assign the states of the particles in the 
GHZ state so that the parities of the outcomes of the measurements are
consistent with the predictions of quantum mechanics.

EPR and Bell put together show the following: If the relevant predictions of 
quantum mechanics are correct, then the theory is non-local. The wealth of 
experimental evidence put together since the 1960s has shown that the
predictions of quantum mechanics are indeed correct, and therefore, quantum
mechanics is a non-local theory.

\section{Non-local Causation?}

As previously mentioned, Griffiths acknowledges the locality given by EPR-Bell, 
but refers to this non-locality as ``ethereal'' and not causal. This begs the
question of what we mean by causality, and when we can conclude that an
interaction is or isn't causal. Until we have a clear account of causality, we 
cannot say whether or not the correlation between the measurements in the GHZ
state is causal.

There are a number of philosophical theories of causation, but the dominant
approach in the philosophy of science is the counterfactual theory of causation
as formulated by~\cite{Lewis_1973}. On a simple telling, this theory states that
$A$ causes $B$ if and only if $B$ would not have occurred if $A$ had not
occurred while holding fixed all other variables. The statistical method of
causal discovery, as formulated by~\cite{Pearl_1995} relies on this theory, 
formulating a counterfactual as a conditional probability. The counterfactual
probability of $B$ given $A$ is the probability that $B$ would have occurred if
$A$ had occurred, while holding fixed all those variables which do not lie along
a causal pathway from $A$ to $B$. This method has been used to great success in
a number of fields, including epidemiology, economics, and computer science.

To see how causal inference is conducted, consider the following example.
Suppose we have a dataset of patients who have been treated with a new drug, and
we want to know whether the drug is effective. We can use the counterfactual
method to estimate the causal effect of the drug on the patients. We can compare
the outcomes of the patients who were treated with the drug to the outcomes of
the patients who were not treated with the drug, while holding fixed all other
variables which may have influenced the outcome of the patients. This is called
a randomized interventional trial, and is thought to detect causal relationships
between variables.

This method of causal inference is based on the underlying assumption of the 
\textit{causal markov condition}, which simply states that statistical
dependencies in a causal system must be captured in the causal graph of that
system as either causal relationships or as a product of a common cause. For
causal inference to be valid, we must be able to construct a causal model of
for which the Markov condition holds in every system, else the implication is
that causal inference may detect spurious relationships between variables which
should not be causally linked~\cite{Geiger_1990}.

Let us consider the GHZ state in the context of this method of causal inference.
If we consider the counterfactual probability of the outcome of the third
particle given the outcomes of the first two particles, we find that the outcome
of the third particle is perfectly correlated with the outcomes of the first two
particles. By preparing the GHZ state specifically we have conditioned on the 
preparation of the state, and so, as in the EPR argument, we reach the following
conlusion. Either there are some other variables to be considered in our causal
model, or the correlation between the measurements in the GHZ state is causal
according to our causal inference. Given Bell's work, there may be no hidden 
latent variable to consider, and so we must conclude that according to our
common understanding of causality, this influence is causal.

We are now left with two options. Either we accept this relationship as causal,
and thus must acccept causal influence that travels faster than the speed of
light, or our dominant approach to causality is flawed. The nature of this flaw
is concerning — we have a clear statistical dependency in our system which is
we cannot account for in our causal model, indicating that we have found a
violation of the Markov condition which underlies causal inference.
As Griffiths points out, the first option is catastrophic
(to special relativity) — however, I contend, so is the latter (to causality).
If the critical underlying assumption of causal reasoning is shown to be
violated by fundamental physics, then we are forced to reconsider our use of 
causal inference in a number of fields, and to reconsider our understanding of
causality itself.

\section{Discussion}

Griffith's claim that the non-locality of quantum mechanics is not catastrophic
is based on the idea that the non-local influences in quantum mechanics are not
causal, meaning that we are not forced to accept any signal propagation faster
than the speed of light. However, this idea neglects to consider the
implications of concluding that the correlation between the measurements in the
GHZ state are non-causal. If we, like Griffiths, wish to preserve locality of
causation, then we must ask ourselves what changes we need to make to our 
understanding of causation.

One possible solution is to conclude that the Markov condition approximately 
holds. Perhaps it's the case that in all macroscopic systems, the Markov
condition is viable, even though it is violated in quantum systems. This
resolves our issue to a certain degree, but it raises the question of what
exactly is the nature of the boundary between the quantum and classical worlds,
and how should when know when it is appropriate to apply causal inference to a
system, questions which require further research.

Another approach might be to turn to another theory of causation which can
handle the exclusion of non-local causation. Process theory, as discussed in
~\cite{SEP-CausalProcesses} is a theory of causation under which variables $A$
ad $B$ are allowed to have a causal relationship only if there is a process or
mechanism through which we understand $A$ to impact $B$. This theory provides
us with a justification of rejecting non-local causation — two space-like 
sepatated particles cannot have a causal mechanism through which they influence
each other if we are continuing to assume non0signalling, and so we can reject
the causal relationship between the measurements in the GHZ state. However, this
approach means that we need to reevaluate our use of counterfactual reasoning in
causal inference, and to develop new methods of causal inference which are
compatible with process theory.

In conclusion, the non-locality of quantum mechanics shown through the 
EPR and Bell arguments poses a fundamental challenge to the assumptions
underlying our dominant theories of causation and causal inference. While we
have theories of causation such as process theory which can handle the exclusion
of non-local causation, the counterfactual approach which underlies
techniques of causal inference and discovery is fundamentally challenged by the
non-locality of quantum mechanics. While Griffiths may consider this challenge 
to be ``non-catastrophic'', I argue that it poses a critical challenge to our
understanding of causality and requires further research in the philosophy of
physics and causal inference.

\bibliography{references}

\end{document}
