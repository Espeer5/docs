\documentclass[11pt, a4paper]{article}
\usepackage[a4paper, left=30mm, right=30mm]{geometry} % margin=2.6cm

\usepackage[utf8]{inputenc}  % allow utf-8 input
\usepackage[T1]{fontenc}     % use 8-bit T1 fonts
\usepackage{ebgaramond}

\usepackage{hyperref}         % hyperlinks
\usepackage{url}              % simple URL typesetting
\usepackage{booktabs}         % professional-quality tables
\usepackage{amsfonts}         % blackboard math symbols
\usepackage{nicefrac}         % compact symbols for 1/2, etc.
\usepackage{microtype}        % microtypography
\usepackage{xcolor}           % colors
\usepackage{bbm}
\usepackage{amsthm}
\usepackage{rotating}
\usepackage{braket}

\hypersetup{                 % setup the hyperref package
    colorlinks=true,
    linkcolor=blue,
    filecolor=blue,
    urlcolor=blue,
    citecolor=blue,
    bookmarks=true,
}

\usepackage{graphicx} % Required for inserting images
\usepackage[ngerman, english]{babel}
\usepackage[iso, ngerman]{isodate}

\usepackage{bbold}
\usepackage{mathtools}
\usepackage{amsmath} % Required for \DeclareMathOperator
\usepackage{nicefrac}
\usepackage{tikz}
\usepackage{subcaption}
\usepackage{centernot} % for the comparison

%\theoremstyle{definition}
\newtheorem{definition}{Definition}
\newtheorem{theorem}{Theorem}

\usepackage{verbatim}

\RequirePackage[T1]{fontenc} 
\RequirePackage[tt=false, type1=true]{libertine} 
\RequirePackage[varqu]{zi4} 
\RequirePackage[libertine]{newtxmath}

\usepackage[round,comma]{natbib}
\bibliographystyle{plainnat}

\newcommand{\monthyeardate}{\ifcase \month \or January\or February\or March\or %
April\or May \or June\or July\or August\or September\or October\or November\or %
December\fi, \number \year} 

\title{HPS/Pl 125: Problem 4}
\author{%
  Edward Speer
  \\
  California Institute of Technology\\
  HPS/Pl 125, WI '25 \\
}
\date{\monthyeardate}

\begin{document}

\maketitle

\noindent \emph{Choose one or two of Maudlin's 8 experiments, and develop an 
argument that either Bohmian mechanics or GRW theory provide a better
explanation of your selected experiments.}
\\\hfill\\
I confess I am enitrely unsettled by a question regarding explanation in quantum
mechanics.I have in the past been quite set on causal explanations, but given 
that I haven't yet been convinced that causality is safe in quantum mechanics,
or that the influences shown between particles in quantum mechanics are causal,
I am not sure exactly which route to take towards discussing explanation in 
quantum mechanics. In such a situation, I would like to fall back to a simpler
model of explanation, like the DN model, but here I am greeted by 2 sets of laws
which are each capable of logically yielding the equal empirical phenomena. It
seems likely that both approaches offer similarly unifying explanations of the
phenomena, so a unificationist account also does not provide a clear answer on
a cursory look. Unsure of where else to turn, I will fall back on a pragmatic
account of explanation, and look for features of the two theories which might
be more appealing or useful in practice when applied to the two experiments.

For the experiment which I think serves to show the greatest contrast
between the two provided explanations, I will choose that of the classic double
slit experiment. To summarize, in the double slit experiment, a particle is
fired at a screen with two slits, and an interference pattern is observed on a
screen behind the slits. The interference pattern is consistent with wave
behavior, but when the particle is fired one at a time, it is observed in
discrete flashes on the screen, as if it were a particle. We never observe a
deviation from the standard predictions of quantum mechanics in this experiment
as it has been performed. 

Let us first address the double slit experiment. Bohmian mechanics offers a 
straightforward explanation. Both the particle and the wave are real entities.
Since the particle is guided by the wave, it is not surprising that the particle
will behave as a particle when observed, and as a wave when not observed. The
wave is a real entity, and so it is not surprising that it will interfere with
itself. Particle-wave duality is a natural consequence of the theory. In GRW, 
however, we find a different explanation. We have a single element of our
ontology, which is the wave. The wave is a real entity, and it is subject to
a stocastic process of collapse. The collapse is a real physical process, which
is incredibly unlikely to occur until the wave becomes entangled with the
macroscopic screen behind the slits. This explanation is pragmatically less
desireable for the reason of ``likelihoods''. In principle, if the slits were
made long enough, or if the experiment were repeated enough times, we should see
collapses before the screens, and we could possibly see a particle fail to
collapse for a short time at the screen. Our observed pattern of wave-particle 
duality as we typically see it is then subject to potential violations. The
pattern we observe is not a necessary pattern, but one which falls out of 
statistical likelihoods with the potential for violation. In Bohmian mechanics,
the wave is always guiding the particle, and there is nothing random about the
process. Every particle will obey the guiding equation, and we don't have to
worry about these random unlikely events outside of our standard predictions or
results. So on pragmatic grounds, we should prefer the explanation from Bohmian
mechanics. This type of reasoning can be expanded to the broader issue of tails
in GRW theory to apply similarly to the EPR experiment as well.


\end{document}