Thanks to the famed Bell-EPR theorem, quantum mechanics is often described as
an inherently non-local theory. The theorem provides us with a mystery — two
particles are entangled in such a way that they must have opposite spins, but
the outcome of each spin measurement cannot be pre-determined before
measurement. As a result, it seems that measuring the spin of one particle
instantly impacts the spin of the other particle, even if they are separated by
vast distances, which seems to violate the principle of locality and make our
fundamental physics mind-bendingly non-local. However, the Bell-EPR methodology
makes one hidden assumption about the outcomes of quantum measurements: that
there is only one outcome per measurement. The Many-Worlds Interpretation (MWI)
of quantum mechanics challenges this assumption by positing that all possible
outcomes of a quantum measurement actually occur, each in its own separate
``world.'' This begs the question: if the MWI is true, is quantum mechanics a
local theory?

In response to questions about locality in quantum mechanics, most physicists
are quick to point to no-signaling theorems, which show that quantum
non-localities cannot be hijacked to send information faster than the speed of
light. As Maudlin puts it, ``One prominent suggestion for\ldots [a] proper and
important sense of `local' has to do with \emph{signaling}. In this sense, a
physical theory is non-local just in case one can specify how to use the physics
to send useful, interpretable signals fast than light'' \citep{Maudlin_2014}.
However, as Marc Lange points out, ``Nevertheless, the left measurement event
apparently helps to cause something to happen on the right (namely, the right
particle acquiring some definite spin component) with no causes in between''
\citep{Lange_2002}. Despite the no-signaling theorem, it seems that quantum
mechanics allows causal influences to propagate faster than the speed of light,
a disturbing prospect for our fundamental understanding of the universe.

In this paper, I will argue that the MWI resolves the locality problem in
quantum mechanics in a way that salvages local causality. I will defend the 
claim that the MWI is a local theory regardless of whether branching events
propagate faster than the speed of light. I will also show that concerns about
locality due to separability for the MWI do not pose a threat to causality, and
argue that the MWI's ability to resolve the locality problem in quantum
mechanics serves as a reason to select the MWI over competing interpretations.

This paper will proceed as follows. I will first present a brief overview of 
the Bell-EPR theorem, and defend why the non-locality it implies is a threat to
our understanding of causality despite the no-signaling result. I will then
present the MWI of quantum mechanics, motivated by a brief description of the
measurement problem it is designed to solve. I will discuss two branch
propagation schemes, the light-cone branching of David Wallace and global
branching, and show why both schemes are compatible with local causality. I will
then describe the issue of separability for the MWI, and show that it does not
pose a threat to local causality. Finally, I will conclude by arguing that the
MWI's ability to resolve the locality problem in quantum mechanics serves as a
strong reson to favor the MWI over other interpretations of quantum mechanics.
