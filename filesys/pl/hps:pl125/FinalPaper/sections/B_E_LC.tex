The Bell-EPR theorem is the quintessential evidence of non-locality in quantum
mechanics. I will begin with a careful presentation of the theorem, so that I
can later show how the MWI undermines the non-locality it implies. 

\subsection{Bell-EPR}

In 1935, Einstein, Podolsky, and Rosen (EPR) published a paper which attempted
to prove that quantum mechanics and the quantum wave function provide an
incomplete description of reality \citep{EPR_1935}. They ask the reader to imagine
two particles entangled in such a way that they must have opposite spins along
some axis $Z$. Prior to any measurement, each particle has a 50\% chance of
being $Z$ spin-up or $Z$ spin-down. When one of the particles is measured using
a Stern-Gerlach apparatus, quantum formalism allows us to immediately predict
the outcome of a measurement performed on the other particle. EPR assume
locality, and conclude that the measurement on the first particle could not have
influenced the second particle, and therefore the outcome of the measurement
must have been fixed from preparation. They conclude that some hidden variable
of each particle not included in the wave function must have encoded the spin of
the particle for the duration of the experiment, and that the wave function 
therefore did not provide a complete description of the state of the particle.

This conclusion was challenged by John Bell in 1964, when he
published a paper which showed that the type of hidden variable demanded by EPR
could not exist \citep{Bell_1964}. Bell's theorem can be most easily grasped by
considering a 3 particle state known as the Greenberger-Horne-Zeilinger (GHZ)
state. The GHZ state is a three-particle entangled state with the following
properties:
\begin{itemize}
    \item If spin is measured along the $x$-axis for all three particles, the 
          number of particles with spin up will always be odd.
    \item If spin is measured along the $z$ axis for two particles, and along
          the $x$-axis for the third particle, the number of particles with spin
          up will always be even.
\end{itemize}
Notice that given the measurement of the first two particles along $x, x$ or
$x, z$, we can immediately predict with certainty the outcome of a measurement
performed on the third particle along $x$, just as we could in the original EPR
case. However, we now should attempt to assign the outcomes of the measurements
prior to measurement, such that the outcomes will always agree with these
predictions. The problem: there is no way to assign these outcomes prior to
measurement to satisfy the constraints. The outcome of the third measurement 
must change depending on the the experimenter's choice of measurement axes for 
the first and second particles measured, and therefore the outcome cannot be
fixed prior to measurement. Bell's theorem shows that no hidden variable theory
can satisfy the constraints of the GHZ state, and therefore that EPR cannot
hold. We therefore obtain a proof by contradiction: If quantum mechanics is local,
then there must be a hidden variable which fixes the outcomes of measurements on
entangled systems from preparation. There may be no such hidden variables, and 
therefore quantum mechanics must be non-local.

\subsection{Non-local Causality}

The Bell-EPR theorem provides us with a puzzling mystery. We have two particles
separated from each other, and simply measuring the state of one particle seems
to have an immediate impact on the state of the other particle across a
space-like separation. Many physicists are happy to accept this non-locality and
dismiss it as ``non-causal''. Take what Griffiths has to say about the subject
for example, ``Causal influences cannot propagate faster than light, but there is
no compelling reason why ethereal ones should not. The influences associated
with the collapse of the wave function are of the latter type, and the fact that
they `travel' faster than light may be surprising, but it is not, after all,
catastrophic.'' But on what grounds can we dismiss this non-locality as not
causal? What is the difference between a causal and a non-causal influence? To
answer this question, we must first define what we mean by a causal influence.

To define a causal influence, I will draw on the dominant theory of causality
in contemporary philosophy, know as the counterfactual theory of causation.
According to this theory, conditioning on all other relevant features of the
environment, an event $A$ is a cause of an event $B$ if and only if, had $B$ no
longer occurred, $A$ would not have occurred either. This definition of
causality holds a place of prominence in the philosophy of causation as well as
in practical scientific applications for inferring causal relationships from
correlational data \citep{Lewis_1973} \citep{Pearl_1995}. This method of causal
inference rests on a foundational assumption known as the causal markov
condition (CMC), which holds that given all causal parents of a variable, the
variable is independent of all of its non-causal ancestors and descendants. This
essentially means that if we hold fixed all potential causes of an event, then
any statistical independence we observe between the event and other variables
must be due to causal relationships \citep{Geiger_1990}. 

Now, we can apply this definition of causality to the Bell-EPR theorem. We
have two particles, $A$ and $B$, which are entangled in such a way that
measuring the state of $A$ will immediately determine the state of $B$. We are
able to condition on the preparation of the particles, and we know that there
are no unobserved confounding variables which could be causing the correlation
between them — this is what Bell's theorem shows us. This means that despite
conditioning on the possible causal parents of the two particles, we still
observe a statistical independence between the two particles. Therefore we have
two options: either the particles are causally related (meaning we have
non-local causation) or fundamental physics demonstrates that the causal markov
condition is false. This is a difficult pill to swallow — either option implies
that we will have to revise our fundamental understanding of causality and 
reckon with unfortunate consequences for our understanding of the universe. As a
result, an alternative that would allow us to avoid this conundrum would be
very welcome. The MWI will provide just such an alternative.
