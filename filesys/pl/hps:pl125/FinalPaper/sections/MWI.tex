Quantum mechanics suffers from a fundamental issue known as the measurement
problem. The measurement problem can be understood as the fundamental puzzle of
how quantum superpositions evolve to definite measurement outcomes. This 
problem is often illustrated by the famous thought experiment of Schrödinger's
cat, in which a cat is placed in a box with a radioactive atom and a vial of
poison. The atom has a 50\% chance of decaying in the next hour, and if it does,
the cat will be poisoned and will die. If the atom does not decay, the cat will
live. The Copenhagen interpretation of quantum mechanics tells us that the atom
will be in a superposition of decayed and not decayed until we open the box and
look inside. Does this mean that the cat is also in a superposition of dead
and alive until we open the box and look? Several interpretation of quantum
mechanics have been proposed to solve the measurement problem, but one of the
most popular is the Many-Worlds Interpretation (MWI), which posits that all
possible outcomes of a quantum measurement actually occur, each in its own
separate ``world.'' In this way, the MWI solves the measurement problem by 
eliminating it — superpositions do not evolve to definite outcomes, because all 
outcomes of a quantum measurement actually occur in separate worlds. For 
Schrödinger's cat, there are therefore some worlds in which the cat is dead, and
some worlds in which the cat is alive, and there is no sense in which one
definite outcome must occur \citep{Wallace_2012}.

If we are to accept the MWI, we need to develop a sharp account of the branching
of worlds. Modern approaches to the MWI have considered branching as a result of
quantum decoherence, a phenomenon in which the wave function of a system
becomes entangled with its environment, causing it to separate into distinct
non-interacting branches. David Wallace describes this approach as giving rise 
to a natural, emergent universe from the structure of ordinary unitary quantum 
mechanics, ``there is an \emph{emergent} branching structure realised by the 
underlying unitary dynamics. In that emergent theory, the configuration space
can be taken to be the space of instantaneous decoherence selected projectors
[of the system's density matrix in the pointer basis]'' \citep{Wallace_2012}.

We are now equipped to think about the MWI in the context of local causality.
Consider the EPR experiment. We know that EPR gives rise to a decoherence when 
each of the anti-correlated particles is entangled with its measurement
apparatus, giving rise to an emergent branching structure as Wallace describes.
Before, the observation of one particle instantly informed us of the state of
the other particle. But how has that changed in the MWI?

\subsection{The Locality of the MWI}

Consider two experimenters, Alice and Bob, who are each measuring the spin of
one particle of an entangled pair in the EPR state, each at a vast space-like
separation from each other. As EPR typically proceeds, Alice measures the spin
of her particle along some axis $Z$, and Bob measures the spin of his particle
along the same axis. As soon as Alice measures the spin of her particle, she
knows beyond a doubt that Bob's particle will have the opposite spin. But in the
MWI, the picture changes significantly. Before measuring her particle, Alice
knows that both possible outcomes of both her measurement and Bob's measurement
will obtain. The only uncertainty Alice has is which world she will find herself
in — the one in which she sees up or the one in which she sees down. Once she 
finds herself in one of those worlds, \emph{nothing has changed about Bob's
state or her knowledge of it.} Alice still knows that Bob's world will branch 
into two branches, one in which Bob's particle is spin up and one in which it is
spin down, regardless of the result Alice has seen. The only knowledge Alice has
gained about Bob is that when she later meets with him to compare notes, she
meet the version of Bob who measured the opposite spin that she did. 

If we now wish to reconsider our causal picture that we previously developed, we
see that the MWI has resolved our concern — we no longer are forced to accept
non-local causality or reject the Markov condition. Given that both outcomes of
each measurement will obtain, we can now happily conclude that there is no 
statistical dependence between the two measurement events, and therefore no
reason to believe that the outcome of one measurement has any causal impact on
the other. The MWI has saved us from the non-local causality of the Bell-EPR
theorem, and allowed us to maintain a local causal picture of the universe.

One should be careful, however, not to prematurely overstate the case, and to
carefully think through any new causal issues that may arise as a result of
the MWI. For example, we have a fundamental question about the branching of the 
universe. When Alice makes her measurement, the universe must branch into two
distinct branches, one in which Alice sees spin up and one in which she sees
spin down. But when this happens, what is happening in Bob's location? Does 
Alice's measurement cause an instantaneous global branching such that as soon as
Alice's particle decoheres, the entire universe branches? If so, does this lead
to a new violation of local causality?

\subsection{Branch Propagation in the MWI}

First, we will consider the favored branch propagation scheme of David Wallace,
known as light-cone branching. In this scheme, the universe branches fully
locally, such that the branching of the universe only moves across space within
the light cone of the event which caused the branching. When Alice makes her
measurement, the universe first branches at the location of the particle
measured. The two branches then diverge across space at a rate no faster than
the speed of light. Alice realizes the result of her measurement in no less than
the time it takes for the light from her measurement device to reach her eyes. 
If Alice wants to tell Bob the result of her measurement, she can only
communicate this information to him at a speed no faster than the speed of
light, not only due to her own limitations, but also because the branching
event itself cannot propagate faster than the speed of light. In this way, the
light-cone branching scheme of the MWI is fully local, and does not violate any
local causal principles.

This type of light cone branching fully and naturally resolves all of our
concerns regarding the locality of branching, but notice that we pay a price for
it. While Wallace likes to describe the MWI as ``just quantum mechanics itself,
taken literally as a description of the universe'' \citep{Wallace_2012}, this
is no longer totally true if we accept the light-cone branching scheme. There is
nothing about unitary quantum mechanics itself or Schrödinger evolution that
requires the propagation of branching events to be limited in this way — it 
requires an additional assumption or postulate about the universe. As a result,
we should also consider other possible branch propagation schemes that could be
consistent with the MWI.

A less straightforwardly local branch propagation scheme, known as global
branching, is a very simple concept — at the time of a measurement event, the
entire universe branches into multiple worlds instantly, regardless of the
distance or connection to the measurement event. In this scheme, when Alice
measures her particle, Bob branches into two distinct worlds, one in which he 
\emph{will} observe spin-up and one in which he \emph{will} observe spin-down.
This leaves us with the unintuitive result that there are temporarily two
identical Bobs who have not yet been differentiated in any way, but will be in 
the future. \cite{Sebens_2018} describe the unintuitive consequences of this
branching scheme, ``the globally-branching view might cause some discomfort. It
implies that observers here on Earth could be (and almost surely are) branching
all the time, without noticing it, due to quantum evolution of systems in the 
Andromeda Galaxy and elsewhere throughout the universe.'' Does this apparently
non-local interaction violate our local causality and undo the good work done by
the MWI in resolving the Bell-EPR non-local causality?

\cite{NeyForthcoming} provides us with a way to think about this issue. Ney 
raises the notion of a Cambridge change, a change that is not intrinsic to an
object itself, but rather a change in extrinsic relationships to the environment
around it. She uses the example of Socrates and his wife Xanthippe, who is
located very far away from Socrates. When Socrates is made to drink hemlock in
prison in Greece, he dies, and therefore Xanthippe becomes a widow instantly
across whatever distance separates them. Despite the spatial separation between
Socrates' death and Xanthippe, there is nothing about this change in  Xanthippe
that is causal or implies action at a distance. Instead, we merely observe a 
descriptive change in Xanthippe's relationship to her environment. In the same
way, when Bob's universe branches as a result of a decoherence in Alice's 
location, no intrinsic change has occurred in Bob's state. Bob remains unchanged,
and his probability of measuring spin up or spin down remains 50\%, just as
before. The only change that has occurred is a descriptive one — Bob is now one
of a pair or multitude of identical Bobs, each of whom will measure different
outcomes. Just as Xanthippe became a widow, Bob has become a member of a
collection of Bobs, but this doesn't imply any sort of causal influence or
action at a distance exerted over Bob. As a result, even in the global branching
scheme, we can maintain a local causal picture of the universe through the MWI.
