In this paper, I have argued that the Many-Worlds Interpretation of quantum
mechanics is a local theory in a way that defeats the EPR-Bell theorem and
allows us to maintain a local causal picture of the universe. I have defended
the claim that the typical picture of quantum mechanics as a non-local theory 
does threaten our understanding of causality, and demonstrated that the MWI
offers us a way out of this predicament, regardless of the branching propagation
scheme we choose to adopt and even when considering the non-separability of 
the quantum state.

A critical result of this argument is that the MWI is unique in its ability to 
resolve the problem of local causality in quantum mechanics. The competitors to 
the MWI, such as spontaneous collapse and Bohmian mechanics, do not offer the
same caveat to the non-locality of quantum mechanics that the MWI does. Since in
these theories, unique results obtain for each measurement, none of these 
options can successfully escape the threats to causality posed by the EPR-Bell 
theorem. How strong of a reason is this to prefer many-worlds over the other 
interpretations? I argue that it increases the unificationist character of the 
MWI. Already the MWI is the most parsimonious interpretation of quantum
mechanics, with the most readily available extensions to relativistic quantum 
mechanics and quantum field theory. Now, we see that the MWI is also the only
interpretation of quantum mechanics that can resolve quantum mechanics with our
understanding of causality. In lieu of experimental evidence to adjudicate
between the competing interpretations, I believe this is a strong reason to
prefer the MWI over its competitors.

One may disagree with this conclusion, and argue that the MWI's ability to
resolve the locality problem is not a strong reason to prefer it over its
competitors. Some readers may not be convinced that local causality in quantum
mechanics poses a broad threat to our understanding of causality and the universe
at large, and therefore may not be convinced that the MWI's resolution of this
issue is significant. However, this is just one of several ways in which the 
MWI provides a more unified account of quantum mechanics than its competitors,
and as these reasons accumulate, it becomes more difficult to dismiss each 
individual reason as unimportant. While one may not be convinced that the MWI's
resolution of the locality problem is a strong reason to prefer it, it at least
adds to mounting evidence that the MWI is a natural and unified interpretation of
the theory.
