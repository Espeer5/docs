\documentclass[11pt, a4paper]{article}
\usepackage[a4paper, left=30mm, right=30mm]{geometry} % margin=2.6cm

\usepackage[utf8]{inputenc}  % allow utf-8 input
\usepackage[T1]{fontenc}     % use 8-bit T1 fonts
\usepackage{ebgaramond}

\usepackage{hyperref}         % hyperlinks
\usepackage{url}              % simple URL typesetting
\usepackage{booktabs}         % professional-quality tables
\usepackage{amsfonts}         % blackboard math symbols
\usepackage{nicefrac}         % compact symbols for 1/2, etc.
\usepackage{microtype}        % microtypography
\usepackage{xcolor}           % colors
\usepackage{bbm}
\usepackage{amsthm}
\usepackage{rotating}
\usepackage{braket}
\usepackage{physics}

\hypersetup{                 % setup the hyperref package
    colorlinks=true,
    linkcolor=blue,
    filecolor=blue,
    urlcolor=blue,
    citecolor=blue,
    bookmarks=true,
}

\usepackage{graphicx} % Required for inserting images
\usepackage[ngerman, english]{babel}
\usepackage[iso, ngerman]{isodate}

\usepackage{bbold}
\usepackage{mathtools}
\usepackage{amsmath} % Required for \DeclareMathOperator
\usepackage{nicefrac}
\usepackage{tikz}
\usepackage{subcaption}
\usepackage{centernot} % for the comparison

%\theoremstyle{definition}
\newtheorem{definition}{Definition}
\newtheorem{theorem}{Theorem}

\usepackage{verbatim}

\RequirePackage[T1]{fontenc} 
\RequirePackage[tt=false, type1=true]{libertine} 
\RequirePackage[varqu]{zi4} 
\RequirePackage[libertine]{newtxmath}

\usepackage[round,comma]{natbib}
\bibliographystyle{plainnat}

\newcommand{\monthyeardate}{\ifcase \month \or January\or February\or March\or %
April\or May \or June\or July\or August\or September\or October\or November\or %
December\fi, \number \year} 

\title{HPS/Pl 125: Problem 6}
\author{%
  Edward Speer
  \\
  California Institute of Technology\\
  HPS/Pl 125, WI '25 \\
}
\date{\monthyeardate}

\begin{document}

\maketitle

\noindent \emph{Belot presents four different ideas about the nature of the wave
function in Bohmian mechanics — multi-field, field, law, or property. Which of
these four options do you think Bohmians should adopt and why?}

I find myself generally disturbed by any ontology which requires that there be
a phyiscal manifestation of what the author refers to as a
``zillion-dimensional'' space for the wave function. I find the idea that the 
configuration space exists outside of physical space to be a bleak prospect — 
how could we gain any epistemic access to such a space, or learn anything about
it? I am afraid we could do no more than simply postulate it, without the
ability to gain or learn anything through doing so. On the other hand, if the
configuration space is the physical space, I find it very difficult to
understand why our experience of the world would be so limited to 3. I see no
justification for claiming our world to be many-dimensional. As a result, I do
not favor the field interpretation.

I similarly find the far-other end of the spectrum to be unpalatable. The idea 
that the wave function is purely nomological and does not in any way represent 
a material structure does not seem realistic to me. We see true patterns of 
interaction that manifest in physical space out of quantum mechanics in the 
double slit experiment, and I feel that if the wave function is removed from
physicality, reduced to nomological status, we lose the ability to explain
these patterns. I am further concerned that the law interpretation introduces
laws of nature that change over time, which seems to me to represent a severe
departure from the way we understand the world to work.

I am left with the multi-field and property interpretations. I have some
concerns about the property interpretation, as I feel that we should either gain
or maintain the full informatic content of the wave function when we posulate an
ontology. It doesn't seem right that we should lose information about the
wave function when we posit an ontology, but the property interpretation seems
to require this. The example given by Belot demonstrates this for the particle
in the box, where any solution corresponding to energy eigenstates of the
particle in the box yields the same history of dispositions, giving equal danger
and equal probability of being found in any region of the box between all of the
solutions. In contrast, standard quantum mechanics would give us more
information with different dangers and different probabilities for different
solutions. I find it very difficult to commit to an ontology in which it seems
we lose some information in the process.

So, all that remains standing is the multi-field interpretation. While this 
approach certainly has its challenges, I found it the most promising of those 
enumerated. The fields postulated here are a bit lacking by comparison to the
fields of classical physics, and the have different motivations, but I find 
these challenges pale in comparison to the issues with the other options 
discussed above. In the multi-field version, we don't get a strange interaction
across configuration space, and we don't have to tie a large ensemble of
particles together across space. We get to connect a field to each particle,
and work with the particle species and degrees of freedom of the particles in 
physical space. This seems to me to be the most promising of the options
discussed by Belot.

\end{document}
