\emph{\textbf{Q:} Could you produce the analogous argument for the same
      conclusion using the state
      $\ket{A} = \ket{z\uparrow}_R\ket{z\downarrow}_L$ in which the spins
      are aligned?}
\hfill \\

Let's attempt to construct the analogous argument for the given state. The
original argument hinges on the fact that knowing the spin of one particle
along some axis allows us to predict with certainty the spin of the other
particle, mathematically shown to hold regardless of the axis along which you
measure or predict the spin. The mathematic proof in question is simply to 
rewrite the z-spin state in terms of the x-spin state and observe the
consequences. So allow us to do the same and rewrite $\ket{A}$ in the x-state.
We know that $\ket{z\uparrow} = \frac{1}{\sqrt{2}}\ket{x\uparrow}+
\frac{1}{\sqrt{2}}\ket{x\downarrow}$, such that
\[\ket{A} = \ket{z\uparrow}_R\ket{z\downarrow}_L = (\frac{1}{\sqrt{2}}
\ket{x\uparrow}_R+\frac{1}{\sqrt{2}}\ket{x\downarrow}_R)(\frac{1}{\sqrt{2}}
\ket{x\uparrow}_L+\frac{1}{\sqrt{2}}\ket{x\downarrow}_L)\]
\[= \frac{1}{2}(\ket{x\uparrow}_R\ket{x\uparrow}_L +
    \ket{x\uparrow}_R\ket{x\downarrow}_L +
    \ket{x\downarrow}_R\ket{x\uparrow}_L +
    \ket{x\downarrow}_R\ket{\downarrow}_L)\]

Now we can see that, in terms of x-spin, the state $\ket{A}$ is an equal
superposition of all possible x-spin states. This means that the EPR argument
doesn't apply - given the x-spin of one particle, we can't predict anything
about the x-spin of the other particle. Why is this?

Note that the issue is not the alignment of the spins. To prove this,
consider instead a superposition of 2 spin-aligned states, and see if the EPR
argument holds.
\[\ket{L} = \frac{1}{\sqrt{2}}\ket{z\uparrow}_R\ket{z\uparrow}_L
          + \frac{1}{\sqrt{2}}\ket{z\downarrow}_R\ket{z\downarrow}_L\]
\[=\frac{1}{\sqrt{2}}(
  (\frac{1}{\sqrt{2}}\ket{x\uparrow}_R+\frac{1}{\sqrt{2}}\ket{x\downarrow}_R)
  (\frac{1}{\sqrt{2}}\ket{x\uparrow}_L+\frac{1}{\sqrt{2}}\ket{x\downarrow}_L) +
  (\frac{1}{\sqrt{2}}\ket{x\uparrow}_R-\frac{1}{\sqrt{2}}\ket{x\downarrow}_R)
  (\frac{1}{\sqrt{2}}\ket{x\uparrow}_L-\frac{1}{\sqrt{2}}\ket{x\downarrow}_L))\]
\[=\frac{1}{\sqrt{2}}\ket{x\uparrow}_R\ket{x\uparrow}_L +
   \frac{1}{\sqrt{2}}\ket{x\downarrow}_R\ket{x\downarrow}_L\]

and in this aligned-spin state, we again are able to predict the spin of one
particle given the spin of the other along any axis. So clearly, it is not
the alignment or anti-alignment of the state that does the work here - it is
instead the entanglement of the state.

Notice that state $\ket{A}$ is a product state - it can be written as a simple
product of 2 z-spin states, meaning that the spins of the 
two particles are independent of each other. Yes, the spins are aligned, but 
this alignment is not due to any entanglement, it is just the way the state was
constructed. This means that the spins of the two particles are not correlated
in any way, and so the EPR argument does not apply to this state.

On the other hand, both $\ket{S}$ and $\ket{L}$ are entangled states. The wave 
functions do not have the same separability and do not factor into a simple 
product of the state of the right particle and the left particle. This means
that the spins of the two particles are correlated in some way, and the nature
of this correlation is what gives rise to the EPR argument.
