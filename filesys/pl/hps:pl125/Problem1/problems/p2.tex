The EPR argument is an attempt to show that quantum mechanics is an incomplete
description of the physical world. The argument proceeds as follows:
\begin{enumerate}
    \item When making measurements of the spin of two entangled particles, the
          results of the measurements are correlated - reading the spin of one 
          particle allows us to instantaneously predict with certainty the spin
          of the other.
    \item No information may travel faster than the speed of light.
    \item If information about the spin of one particle is not transmitted to
          the other particle, then the results of the measurements on the two
          particles must have been determined before the measurements were made.
    \item If the measurements were determined before the measurements were made,
          but couldn't be ascertained from the wave function, then the wave
          function does not provide a complete description of the physical
          state of the system.
    \item If wave functions do not provide a complete description of the
          physical state of a system, then quantum mechanics does not provide a 
          complete description of the physical world, and is therefore
          incomplete.
\end{enumerate}

Notice that the above argument hinges on a key detail. If the wave function
provides a complete description of the physical state of a system, then the 
EPR argument fails, and we may be forced to conclude that the measurement of one
spin somehow transmits to the entangled particle the information about the
measurement instantaneously - spooky action at a distance. If, on the other hand,
the wave function does not provide a complete description of the system, then
there must be some other hidden variable that determines the results of the
measurement. Therefore, the presence of a hidden variable gains deep
significance in attempting to determine whether or not quantum mechanics demands
spooky action at a distance. Therefore, someone (Bell) needs to come along to shed light on
the possibility of a hidden variable...
