\documentclass[11pt, a4paper]{article}
\usepackage[a4paper, left=30mm, right=30mm]{geometry} % margin=2.6cm

\usepackage[utf8]{inputenc}  % allow utf-8 input
\usepackage[T1]{fontenc}     % use 8-bit T1 fonts
\usepackage{ebgaramond}

\usepackage{hyperref}         % hyperlinks
\usepackage{url}              % simple URL typesetting
\usepackage{booktabs}         % professional-quality tables
\usepackage{amsfonts}         % blackboard math symbols
\usepackage{nicefrac}         % compact symbols for 1/2, etc.
\usepackage{microtype}        % microtypography
\usepackage{xcolor}           % colors
\usepackage{bbm}
\usepackage{amsthm}
\usepackage{rotating}
\usepackage{braket}

\hypersetup{                 % setup the hyperref package
    colorlinks=true,
    linkcolor=blue,
    filecolor=blue,
    urlcolor=blue,
    citecolor=blue,
    bookmarks=true,
}

\usepackage{graphicx} % Required for inserting images
\usepackage[ngerman, english]{babel}
\usepackage[iso, ngerman]{isodate}

\usepackage{bbold}
\usepackage{mathtools}
\usepackage{amsmath} % Required for \DeclareMathOperator
\usepackage{nicefrac}
\usepackage{tikz}
\usepackage{subcaption}
\usepackage{centernot} % for the comparison

%\theoremstyle{definition}
\newtheorem{definition}{Definition}
\newtheorem{theorem}{Theorem}

\usepackage{verbatim}

\RequirePackage[T1]{fontenc} 
\RequirePackage[tt=false, type1=true]{libertine} 
\RequirePackage[varqu]{zi4} 
\RequirePackage[libertine]{newtxmath}

\usepackage[round,comma]{natbib}
\bibliographystyle{plainnat}

\newcommand{\monthyeardate}{\ifcase \month \or January\or February\or March\or %
April\or May \or June\or July\or August\or September\or October\or November\or %
December\fi, \number \year} 

\title{HPS/Pl 125: Problem 3}
\author{%
  Edward Speer
  \\
  California Institute of Technology\\
  HPS/Pl 125, WI '25 \\
}
\date{\monthyeardate}

\begin{document}

\maketitle

\noindent \emph{Consider the Gaussian wave packet given by
\[\psi(x)=(\frac{1}{\pi\omega^2})^{\frac{1}{4}}e^{\frac{-x^2}{2\omega^2}+ikx}\]}

\emph{* Note that all odd integrals dissapear and all integrals shown are taken
over $-\infty$ to $\infty$.}

\section{Part a} \emph{Compute the expectation values of $\hat{p}$ and
$\hat{H}$ on this state}
\\ \hfill \\
The expectation value of an operator $\hat{A}$ on a state $\psi$ is given by
\[\braket{\hat{A}} = \int \psi^*(x)\hat{A}\psi(x)dx\] over all space. Thus:

\[\braket{\hat{p}} = \int \psi^*(x)\hat{p}\psi(x)dx\]
\[= -i\hbar(\frac{1}{\pi\omega^2})^{\frac{1}{2}}\int e^{\frac{-x^2}{2\omega^2}-ikx} \frac{\partial}{\partial x}[e^{\frac{-x^2}{2\omega^2}+ikx}]dx\]
\[=-i\hbar(\frac{1}{\pi\omega^2})^{\frac{1}{2}}\int e^{\frac{-x^2}{2\omega^2}}(\frac{-x}{\omega^2}+ik)dx\]
\[= -i^2\hbar k(\frac{1}{\pi\omega^2})^{\frac{1}{2}}(\pi\omega^2)^{\frac{1}{2}}\]
\[\boxed{=\hbar k}\]

\[\braket{\hat{H}} = \int \psi^*(x)\hat{H}\psi(x)dx\]
\[= \frac{-\hbar^2}{2m}(\frac{1}{\pi\omega^2})^\frac{1}{2}\int e^{\frac{-x^2}{2\omega^2}-ikx}\frac{\partial^2}{\partial x^2}[e^{\frac{-x^2}{2\omega^2}+ikx}]dx\]
\[=\frac{-\hbar^2}{2m}(\frac{1}{\pi\omega^2})^\frac{1}{2}\int e^{\frac{-x^2}{\omega^2}}(\frac{x^2}{\omega^4}-\frac{1}{\omega^2}-\frac{2ikx}{\omega^2}-k^2)dx\]
\[= \frac{-\hbar}{2m}(\frac{1}{2\omega^2}-\frac{1}{\omega^2}-k^2)\]
\[\boxed{=\frac{\hbar^2}{2m}(\frac{1}{2\omega^2}+k^2)}\]

\section{Part b} \emph{Collapse the wave function by multiplying it by a
gaussian given as \[(\frac{\omega^2}{\sigma^2}+1)^{\frac{1}{4}}e^{\frac{-x^2}{2\sigma^2}}\]
Compute the expectation values of $\hat{p}$ and $\hat{H}$ on this state}
\\ \hfill \\
\[\braket{\hat{p}} = \int (\frac{\omega^2}{\sigma^2}+1)^{\frac{1}{4}}e^{\frac{-x^2}{2\sigma^2}}\psi^*(x)\hat{p}(\frac{\omega^2}{\sigma^2}+1)^{\frac{1}{4}}e^{\frac{-x^2}{2\sigma^2}}\psi(x)dx\]
\[=-i\hbar(\frac{1}{\pi\sigma^2}+\frac{1}{\pi\omega^2})^\frac{1}{2}\int e^{-(\frac{1}{2\sigma^2}+\frac{1}{2\omega^2})x^2-ikx}\frac{\partial}{\partial x}[e^{-(\frac{1}{2\sigma^2}+\frac{1}{2\omega^2})x^2+ikx}]dx\]
\[=-i\hbar(\frac{1}{\pi\sigma^2}+\frac{1}{\pi\omega^2})^\frac{1}{2}\int e^{-(\frac{1}{\sigma^2}+\frac{1}{\omega^2})x^2}(\frac{x}{\sigma^2}-\frac{x}{\omega^2}+ik)dx\]
\[=-i\hbar(\frac{1}{\pi\sigma^2}+\frac{1}{\pi\omega^2})^\frac{1}{2}(ik)(\frac{\pi}{\frac{1}{\sigma^2}+\frac{1}{\omega^2}})^{\frac{1}{2}}\]
\[=\boxed{\hbar k}\]

\[\braket{\hat{H}} = \int (\frac{\omega^2}{\sigma^2}+1)^{\frac{1}{4}}e^{\frac{-x^2}{2\sigma^2}}\psi^*(x)\hat{H}(\frac{\omega^2}{\sigma^2}+1)^{\frac{1}{4}}e^{\frac{-x^2}{2\sigma^2}}\psi(x)dx\]
\[=\frac{-\hbar^2}{2m}(\frac{1}{\pi\sigma^2}+\frac{1}{\pi\omega^2})^\frac{1}{2}\int e^{-(\frac{1}{2\sigma^2}+\frac{1}{2\omega^2})x^2-ikx}\frac{\partial^2}{\partial x^2}[e^{-(\frac{1}{2\sigma^2}+\frac{1}{2\omega^2})x^2+ikx}]dx\]
\[=\frac{-\hbar^2}{2m}(\frac{1}{\pi\sigma^2}+\frac{1}{\pi\omega^2})^\frac{1}{2}\int e^{-(\frac{1}{\sigma^2}+\frac{1}{\omega^2})x^2}(k^2+(\frac{1}{2\sigma^2}+\frac{1}{2\omega^2})-2ik(\frac{x}{2\sigma^2}+\frac{x}{2\omega^2})+(\frac{x}{2\sigma^2}+\frac{x}{2\omega^2})^2)dx\]
\[=\frac{-\hbar^2}{2m}(\frac{1}{\pi\sigma^2}+\frac{1}{\pi\omega^2})^\frac{1}{2}(\frac{\pi}{\frac{1}{\sigma^2}+\frac{1}{\omega^2}})^{\frac{1}{2}}(-k^2-\frac{1}{2\sigma^2}-\frac{1}{2\omega^2})\]
\[\boxed{=\frac{\hbar}{2m}(\frac{1}{2\sigma^2}+\frac{1}{2\omega^2}+k^2)}\]

\section{Part c} \emph{Explain why the collapse law multiplies the $\psi$ by a
Gaussian instead of just replacing it with a gaussian.}
\\ \hfill \\
We see by comparing the expectation of the momentum operator from parts a and b
that multiplying by a gaussian preserves the expectation value of the momentum
operator. If the expectation value of momentum were to change, the collapse law
would be unphysical and create a discontinuity in both momentum and energy
space. Simply replacing the wavefunction with a gaussian would not preseve the 
expectation value of momentum and would create a discontinuity in momentum space.

\section{Part d} \emph{Explain why $\sigma$ should not be made arbitrarily
small.}
\\ \hfill \\
Compare the expectation of the energy operator from parts a and b. We see that
the system has increased in the expectation of energy by a gain which is given 
by $\frac{\hbar}{2m}(\frac{1}{2\sigma^2})$. If $\sigma$ were to be made
arbitrarily small, this gain would tend towards infinity, meaning that the
system would have a sudden infinite spike in energy. This is unphysical and
would violate observation. In other words, there is a bound in the precision of
localization of a particle for this type of collapse law in order to avoid a
system of infinite energy.

\end{document}