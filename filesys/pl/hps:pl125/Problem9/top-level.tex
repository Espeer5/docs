\documentclass[11pt, a4paper]{article}
\usepackage[a4paper, left=30mm, right=30mm]{geometry} % margin=2.6cm

\usepackage[utf8]{inputenc}  % allow utf-8 input
\usepackage[T1]{fontenc}     % use 8-bit T1 fonts
\usepackage{ebgaramond}

\usepackage{hyperref}         % hyperlinks
\usepackage{url}              % simple URL typesetting
\usepackage{booktabs}         % professional-quality tables
\usepackage{amsfonts}         % blackboard math symbols
\usepackage{nicefrac}         % compact symbols for 1/2, etc.
\usepackage{microtype}        % microtypography
\usepackage{xcolor}           % colors
\usepackage{bbm}
\usepackage{amsthm}
\usepackage{rotating}
\usepackage{braket}
\usepackage{physics}

\hypersetup{                 % setup the hyperref package
    colorlinks=true,
    linkcolor=blue,
    filecolor=blue,
    urlcolor=blue,
    citecolor=blue,
    bookmarks=true,
}

\usepackage{graphicx} % Required for inserting images
\usepackage[ngerman, english]{babel}
\usepackage[iso, ngerman]{isodate}

\usepackage{bbold}
\usepackage{mathtools}
\usepackage{amsmath} % Required for \DeclareMathOperator
\usepackage{nicefrac}
\usepackage{tikz}
\usepackage{subcaption}
\usepackage{centernot} % for the comparison

% \DeclareMathOperator{\Tr}{Tr} % trace operation

%\theoremstyle{definition}
\newtheorem{definition}{Definition}
\newtheorem{theorem}{Theorem}

\usepackage{verbatim}

\RequirePackage[T1]{fontenc} 
\RequirePackage[tt=false, type1=true]{libertine} 
\RequirePackage[varqu]{zi4} 
\RequirePackage[libertine]{newtxmath}

\usepackage[round,comma]{natbib}
\bibliographystyle{plainnat}

\newcommand{\monthyeardate}{\ifcase \month \or January\or February\or March\or %
April\or May \or June\or July\or August\or September\or October\or November\or %
December\fi, \number \year} 

\title{HPS/Pl 125: Problem 9}
\author{%
  Edward Speer
  \\
  California Institute of Technology\\
  HPS/Pl 125, WI '25 \\
}
\date{\monthyeardate}

\begin{document}

\maketitle

\noindent \emph{Explain the incoreherence problem with respect to Everettian 
quantum mehcanics, and either defend what you take to be the best solution, or
argue that the incoherence problem cannot be solved.}
\\ \hfill \\
The many-worlds interpretation of quantum mechanics fundamentally changes the
way we must think about probabilities. In the Copenhagen interpretation, we have
only a single outcome of a measurement, and assign probabilities to the possible
outcomes. This is in keeping with our classical intuition about probabilities
(though, as Wallace goes to great lengths to point out, this ``classical'' 
understanding of probability is itself subject to tremendous issues). In the 
many-worlds interpretation, however, we have a branching of the universe in
which \emph{all} possible outcomes of our measurement occur. This is a radical
departure from our classical intuition about probability — if all possible
measurements obtain, what does it mean to say that one outcome is more likely
than another?

The objection raised above is the \emph{incoherence problem}. Why should we
assign probabilities, and particularly different probabilities, to outcomes
that all actually occur? If many-worlds is correct, then we must have some 
justification for assigning probabilities to outcomes that are all realized
according to the Born rule.

The most compelling option for solving the incoherence problem is that of
non-frequentist, non-primitivist rationalism. This approach avoids critical
issues that can be found in frequentism and primitivism. 

Frequentism is inherently flawed as a solution to the incoherence problem,
because it is fundamentally incompatible with the many-worlds interpretation.
Each outcome of a measurement obtains in a different branch, so what does it
mean for an outcome to be more frequent? We need to fall back on some notion of
branch counting over long term frequencies, but this is unsatisfying. Our
quantum algorithm tells us to assign weights to different branches in the case
of a single measurement. On this single measurement, both outcomes will occur, 
and it seems clear to me that without a radical move to understand frequentist
probabilities in an unintuituve way, we cannot justify using these weights as
probabilities through frequentism.

Primitivism is also unsatisfying, for reasons more pragmatic than technical.
Primitivism is the view that probabilities are primitive, a law of nature
that is accepted alongside Everettian quantum theory. This is a non-starter —
quantum theory has dynamics that are well understood and precise, and it is
not clear that we should have to accept unmotivated primitive probabilities.
With the many-worlds interpretation, we have a precise way to assign weights to
branches, and a clear physical account of what occurs in a measurement. We 
should therefore resist the urge to take as primitive these probabilities as
long as it is possible to avoid it by making progress considering the physical
consequences of the theory.

The rationalist approach avoids these issues by taking probabilities to be
derived from the credences of rational agents. An agent has an uncertainty 
about which quantum branch they will find themselves on following a quantum 
measurement. Since they are rational, they must apportion their credences about
which branch they will end up on in a way that maximizes their utility. Now we
have a setup to work with. Since the rational agent should apportion their
credences in accordance with true probabilities, if we can derive from decision
theory plus quantum theory that the rational agent should assign credences in a
particular way that matches the Born rule, then we have a solution to the
incoherence problem. Wallace provides an example of such a proof in the reading,
and there are other examples of candidate proofs given in the literature. 

While these proofs aren't without their issues (Wallace's proof, for example, 
requires one to make heavy assumptions about the probability space of the
agent's credences), they are the best candidate solution to the incoherence
problem. The rationalist approach makes true progress by deriving the Born rule
from therefore should be pursued as far as it remains viable.

\end{document}
