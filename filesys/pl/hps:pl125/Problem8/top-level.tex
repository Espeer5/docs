\documentclass[11pt, a4paper]{article}
\usepackage[a4paper, left=30mm, right=30mm]{geometry} % margin=2.6cm

\usepackage[utf8]{inputenc}  % allow utf-8 input
\usepackage[T1]{fontenc}     % use 8-bit T1 fonts
\usepackage{ebgaramond}

\usepackage{hyperref}         % hyperlinks
\usepackage{url}              % simple URL typesetting
\usepackage{booktabs}         % professional-quality tables
\usepackage{amsfonts}         % blackboard math symbols
\usepackage{nicefrac}         % compact symbols for 1/2, etc.
\usepackage{microtype}        % microtypography
\usepackage{xcolor}           % colors
\usepackage{bbm}
\usepackage{amsthm}
\usepackage{rotating}
\usepackage{braket}
\usepackage{physics}

\hypersetup{                 % setup the hyperref package
    colorlinks=true,
    linkcolor=blue,
    filecolor=blue,
    urlcolor=blue,
    citecolor=blue,
    bookmarks=true,
}

\usepackage{graphicx} % Required for inserting images
\usepackage[ngerman, english]{babel}
\usepackage[iso, ngerman]{isodate}

\usepackage{bbold}
\usepackage{mathtools}
\usepackage{amsmath} % Required for \DeclareMathOperator
\usepackage{nicefrac}
\usepackage{tikz}
\usepackage{subcaption}
\usepackage{centernot} % for the comparison

% \DeclareMathOperator{\Tr}{Tr} % trace operation

%\theoremstyle{definition}
\newtheorem{definition}{Definition}
\newtheorem{theorem}{Theorem}

\usepackage{verbatim}

\RequirePackage[T1]{fontenc} 
\RequirePackage[tt=false, type1=true]{libertine} 
\RequirePackage[varqu]{zi4} 
\RequirePackage[libertine]{newtxmath}

\usepackage[round,comma]{natbib}
\bibliographystyle{plainnat}

\newcommand{\monthyeardate}{\ifcase \month \or January\or February\or March\or %
April\or May \or June\or July\or August\or September\or October\or November\or %
December\fi, \number \year} 

\title{HPS/Pl 125: Problem 8}
\author{%
  Edward Speer
  \\
  California Institute of Technology\\
  HPS/Pl 125, WI '25 \\
}
\date{\monthyeardate}

\begin{document}

\maketitle

\noindent \emph{Wallace mentions an objection to Bohmian mechanics originally
due to David Deutsch: because Bohmian mechanics includes a wave function
evolving via the Schrodinger equation (and that's all the many-worlds
interpretation says that there is), Bohmian mechanics includes a vast number of
parallel universes (adding particles does nothing to remove these parallel
universes). Thinking about the ontology of Bohmian mechanics and the many-worlds
interpretation, how do you think a Bohmian could best respond to this
objection?}
\\ \hfill \\
The many-worlds of the Everett interpretation stem from the need to resolve a
basic issue. The quantum algorithm gives us a probability amplitude for each
possible outcome of a quantum computation. However, the result of any
measurement made is deterministic. For reasons we've covered at length, we think
that quantum mechanics must be a $\psi$-ontic theory (where the wave function is
a real element of the ontology). This means that \emph{something} must happen to
transition from the probability amplitude to a definite outcome since both
reflect the true state of the world. The many-worlds interpretation resolves
this by \emph{avoiding} resolving it — simply conclude that the definite
outcomes all occur, so that the only thing we need to be real is the probability
almplitude (or, more precisely, the wave function that defines it). In this way,
subscribers to the many-worlds interpretation say that the many-worlds
\emph{emerge} from standard wave function time evolution. Then, by this account,
any interpretation that includes a wave function evolving in time by the
schrodinger equation without an explicit collapse law should also give rise to
the same emergent many-worlds.

Bohmian mechanics is such an interpretation. The wave function evolves in time
via the schrodinger equation, but with a key difference from the standard 
interpretation. Under Bohmian mechanics, there is both a wave function and a
particle in the fundamental ontology. The wave function evolves in accordance
with the schrodinger equation as normal, but in addition, the particles have
definite positions and velocities at each moment in time. The particles move
according to the guidance equation, which is a deterministic equation that
depends on the wave function. Put precisely, the world according to Bohmian
mechanics consists of a set of particles which move around with definite
positions with velocities determined by the wave function.

The determinism here is what allows us to avoid the objection mentioned from the
many-worlds interpretation. The particles have a definite position at each
moment in time, and follow deterministic trajectories. Put another way, there is
a single physical world defined by the particles' positions, which are fully
determined by the specification of the universal wave function and the initial
conditions of the particles. Multiple worlds do \emph{not} emerge from 
schrodinger evolution. The Schdrodinger evolution acts on the wave function,
which plays a guiding role in the particles' motion, but it is the motion of the
particles that defines the physical world.

As an example, consider the case of entanglements that are often used to
showcase the many-worlds interpretation. In the case of Schrodingers cat, 
proponents of many-worlds claim that we see a branching of worlds. In one world,
the cat is alive, while in the other, the cat is dead. Under Bohmian mechanics,
however, we will see that this branching is not necessery. Yes the wave function
will decohere into two distinct packets, one with a living cat and one with a 
dead cat, but in the physical world, the cat willl always be either alive or
dead at any given time. Expressed in a simplified manner, it was always
determined from the initial conditions of the cat and the experimental setup
that the cat would live or die, meaning that the cat follows only one trajectory
given by the decohered wave function. 

\end{document}
