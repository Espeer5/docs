\documentclass[11pt, a4paper]{article}
\usepackage[a4paper, left=30mm, right=30mm]{geometry} % margin=2.6cm

\usepackage[utf8]{inputenc}  % allow utf-8 input
\usepackage[T1]{fontenc}     % use 8-bit T1 fonts
\usepackage{ebgaramond}

\usepackage{hyperref}         % hyperlinks
\usepackage{url}              % simple URL typesetting
\usepackage{booktabs}         % professional-quality tables
\usepackage{amsfonts}         % blackboard math symbols
\usepackage{nicefrac}         % compact symbols for 1/2, etc.
\usepackage{microtype}        % microtypography
\usepackage{xcolor}           % colors
\usepackage{bbm}
\usepackage{amsthm}
\usepackage{rotating}
\usepackage{braket}
\usepackage{physics}

\hypersetup{                 % setup the hyperref package
    colorlinks=true,
    linkcolor=blue,
    filecolor=blue,
    urlcolor=blue,
    citecolor=blue,
    bookmarks=true,
}

\usepackage{graphicx} % Required for inserting images
\usepackage[ngerman, english]{babel}
\usepackage[iso, ngerman]{isodate}

\usepackage{bbold}
\usepackage{mathtools}
\usepackage{amsmath} % Required for \DeclareMathOperator
\usepackage{nicefrac}
\usepackage{tikz}
\usepackage{subcaption}
\usepackage{centernot} % for the comparison

% \DeclareMathOperator{\Tr}{Tr} % trace operation

%\theoremstyle{definition}
\newtheorem{definition}{Definition}
\newtheorem{theorem}{Theorem}

\usepackage{verbatim}

\RequirePackage[T1]{fontenc} 
\RequirePackage[tt=false, type1=true]{libertine} 
\RequirePackage[varqu]{zi4} 
\RequirePackage[libertine]{newtxmath}

\usepackage[round,comma]{natbib}
\bibliographystyle{plainnat}

\newcommand{\monthyeardate}{\ifcase \month \or January\or February\or March\or %
April\or May \or June\or July\or August\or September\or October\or November\or %
December\fi, \number \year} 

\title{HPS/Pl 125: Problem 10}
\author{%
  Edward Speer
  \\
  California Institute of Technology\\
  HPS/Pl 125, WI '25 \\
}
\date{\monthyeardate}

\begin{document}

\maketitle

\noindent \emph{Consider Maudlin's experiment 6, the Mach-Zender interferometer.
                In the many-worlds interpretation, should we say there is a
                world in which the electron travels along the upper path and 
                another in which the electron travels along the lower path?}
\\ \hfill \\

In the Mach-Zender interferometer experiment as Maudlin presented it on pages 
23—24, we have the following setup: a beam of x-spin up electrons is sent
through a z-oriented Stern-Gerlach. Since x-spin up is an equal superposition of
z-spin up and z-spin down, the beam will be split into two beams. One beam will
be z-spin up, sent up from the z-axis, and the other will be z-spin down, sent
down from the z-axis. Each of these two beams has definite z-spin and therefore
is an equal superposition of x-spin up and a-spin down. It seems we have
destroyed the x-spin information contained in the original beam. However, if we
then reflect each beam back towards each other and recombine them, then send the
resulting single beam through an x-oriented Stern-Gerlach, we will see that the
recombined beam is entirely x-spin up. This is a mystery; along each path, we
can verify that there is an equal superposition of x-spin up and x-spin down 
(no x-spin information), yet somehow the information about the beam preparation
is transmitted through the mechanism as a whole.

In most interpretations of quantum mechanics, this issue is resolved by adopting
a $\psi$-ontic view of the wavefunction. In this view, the wavefunction is a
real physical object that describes the state of the system. The wavefunction
exists along each path of the Mach-Zender interferometer (regardless of whether
the particle does or not) and thus the wavefunction is able to carry the
information about the beam preparation through the mechanism as a whole. This
approach holds true for the many-worlds interpretation as well, as the
x-spin up information is carried by the wavefunction and the result predicted
by ordinary time evolution, which functions as the full specification of the
many-worlds interpretation. However, under the many-worlds interpretation, we
end up with an entirely different mystery which arises from this experiment.

Under the many-worlds interpretation, we know that certain superpositions of 
states result in a branching of the universe. In the classic Schroedinger's cat
experiment, the cat is in a superposition of alive and dead states, which 
manifests as a branching of the universe into two worlds: one in which the cat
is alive and one in which the cat is dead. In the Mach-Zender interferometer
experiment, we have a similar situation. The electron is in a superposition of
travelling along the upper path and travelling along the lower path. At a first
glance, it appears that this superposition should result in a branching of the
into two worlds just as the cat experiment does. However, on this view, we
encounter a significant problem. In the many-worlds interpretation, the
different worlds that result from a branching are supposed to be orthogonal to
each other. This means that the worlds are not able to interact with each other
in any way. If this is not the case, then the many-worlds interpretation
wouldn't solve the measurement problem — it is possible that I could still see
macroscopic superpositions of states in my everyday life if my world interacted
with other worlds that resulted from a branch. However, in the Mach-Zender
interferometer experiment, the two worlds that would result from the branching
of the electron travelling along the upper path and the electron travelling
along the lower path are not orthogonal to each other. In fact, they are
entangled with each other and the recombination of the two beams shows that
they must interact. Therefore we \emph{cannot} conclude that there is a
branching of worlds at the beam-splitter if we want many-worlds to remain
viable.

Then clearly, with this in mind, we must reconsider when branching happens in
the many-worlds interpretation. What is it about Schroedinger's cat that causes
branching while the Mach-Zender interferometer does not at the point of the beam
splitter (Note that we still need branching to happen when we measure the final
output of the interferometer)? The best answer available is that branching is
not a result of simple superpositions of states in the wavefunction, but rather
a result of the decoherence of the wavefunction. Decoherence is a process by
which the wavefunction of a system becomes entangled with the wavefunction of
its environment and separates into sufficiently separated components to be
considered separate worlds. In the case of the Mach-Zender interferometer, the
wavefunction of the electron becomes entangled with the wavefunction of the
environment only after passing through the full apparatus and being measured
by a measuring device at the end of the experiment. Thus the wavefunction only
decoheres at the end of the experiment, when the electron is measured. This
means that branching only occurs at the end of the experiment, and should not
be considered to have happened at the beam-splitter. While not entirely
unproblematic, this approach allows us to retain the usefulness of the
many-worlds interpretation while still correctly generating the results of the
Mach-Zender interferometer experiment.

\end{document}
